\chapter{Sobre la Bibliografía}
\label{cap:AnexoA}

En los anexos se incluirá, de modo opcional, material suplementario que podrá consistir en manuales de usuario, listados seleccionados de código fuente, esquemas, planos y en general aquel contenido que complementa a la memoria. Se recomienda que no sean excesivamente voluminosos, aunque su extensión no está sometida a la regulación por normativa, ya que esta afecta únicamente al texto principal de la memoria.

En esta plantilla hemos decidido incluir dos anexos. En el primero de ellos se hacen algunos comentarios adicionales sobre la bibliografía. En el segundo se aporta una breve introducción a \LaTeX{} cuya información puede servir de ejemplo de inclusión de ciertos elementos en la preparación de la memoria del TFG.

Todo el material de terceros se debe citar convenientemente sin contravenir los términos de las licencias de uso y distribución de dicho material. Esto se extiende al uso de diagramas y cualquier elemento gráfico. El incumplimiento de la legislación vigente en materia de protección de la propiedad intelectual es responsabilidad exclusiva del autor, independientemente de la cesión de derechos que este haya convenido.\footnote{\url{https://www.uclm.es/areas/biblioteca/encuentra-informacion/perfiles/alumno/antiplagio}}

La sección de \emph{Bibliografía}, que si se prefiere se puede titular \emph{Referencias}, incluirá un listado ordenado preferentemente por orden alfabético (primer apellido del autor principal), con todas las obras citadas en el texto. En la lista de referencias se especificará para cada obra: 
\begin{itemize}
	\item autores, 
	\item título, 
	\item editorial, y 
	\item año de publicación.
\end{itemize} 

Este formato se conseguirá en \LaTeX{} mediante el uso del estilo estándar \texttt{plain} o cualquier otro derivado con estilo de citación numérica. En algunas titulaciones se obliga a una ordenación por orden de cita en el texto. Este resultado se obtiene con Bib\TeX{} mediante los estilos estándar \texttt{ieeetr} (estilo para los IEEE \emph{transactions}) y \texttt{unsrt} (estilo \emph{unsorted}).\footnote{\url{https://www.wikiwand.com/es/articles/BibTeX}}

Es muy importante tener presente que en esta sección solo se debe incluir las referencias bibliográficas citadas expresamente en el documento. Si se desea incluir fuentes consultadas, pero no citadas, se puede confeccionar con ellas una sección denominada \emph{Material de consulta}, aunque estas referencias se pueden incluir opcionalmente a lo largo del documento como notas a pie de página.

En las titulaciones técnicas se empleará estilo de citación numérico con el número de la referencia entre corchetes. La cita podrá incluir el número de página concreto de la referencia que se desea citar. El uso correcto de la citación implica dejar claro al lector cuál es el texto, material o idea citado. .

Cuando se desee incluir referencias a páginas genéricas de Internet sin mención expresa a un artículo con título y autor definido, dichas referencias se pueden incluir como notas al pie de página o como un apartado de fuentes de consulta dedicado a \emph{Direcciones de Internet}. Por el contrario, los documentos electrónicos publicados en Internet se pueden incluir en la sección de Bibliografía empleando el tipo de entrada \texttt{misc} en el fichero \texttt{.bib} con el comando \texttt{url} (como se muestra en la bibliografía de ejemplo distribuida con este documento). Observarás que el campo \texttt{note} se emplea para añadir información adicional como la fecha de la última consulta de fuentes publicadas en Internet, y para la inclusión del DOI de algunas obras para su rápida recuperación.\footnote{Esta estrategia necesita adaptación cuando se emplea un estilo de citación autor-año (p.ej., \texttt{natbib}).}







