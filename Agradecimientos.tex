\ifspanish
	\selectlanguage{spanish}
\else
	\selectlanguage{english}
\fi

% -------------------------
%
% AGRADECIMIENTOS (recomendable máx. 1 pág.)
%
% -------------------------
\cleardoublepage
\phantomsection % OJO: Necesario con hyperref

\chapter*{Agradecimientos} % Opción con * para que no aparezca en TOC ni numerada
\addcontentsline{toc}{chapter}{Agradecimientos} % Añade al TOC.

%OJO: Editar
 Aunque es un apartado opcional, haremos bueno el refrán \emph{<<es de bien nacidos, ser agradecidos>>} si empleamos este espacio como un medio para agradecer a todos los que, de un modo u otro, han hecho posible que el trabajo realizado \emph{llegue a buen puerto}. Esta sección es ideal para agradecer a directores, profesores, mentores, familiares, compañeros, amigos, etc. 
 
 Estos agradecimientos pueden ser tan personales como se desee e incluir anécdotas y chascarrillos, pero \emph{nunca deberían ocupar más de una página}.

\makeatletter		
\begin{flushright}
	\vspace{1,5cm}
	\textit{\@autor}\\
	\@lugarDef, \@yearDef
\end{flushright}
\makeatother