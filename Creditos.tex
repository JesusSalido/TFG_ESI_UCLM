% -------------------------
%
% CRÉDITOS
%
% -------------------------
% EDITAR: El autor puede elegir el tipo de licencia que desee para distribuir su TFG que puede variar con respecto a la de este documento en el que si permitimos obra derivada para que no surjan dudas sobre la reutilización del material.
% Este comando permite una gran flexibilidad y la ventaja de no depender de paquetes externos.
% Esta es una página reservada para señalar información relativa a los derechos de autor y la licencia de distribución y uso del documento. Esta página debería ser aprovechada también para informar de cualquier tipo de cesión de los derechos anteriormente citados. El autor del TFG debe tener presente que el incumplimiento de la legislación vigente en materia de protección de la propiedad intelectual es de su exclusiva responsabilidad independientemente de la cesión de derechos que este haya convenido para su obra ya que no son objeto de cesión aquellos derechos de los que no se es poseedor.

\creditos{Este documento se distribuye con licencia CC BY-NC-SA 4.0. El texto completo de la licencia puede obtenerse en \url{https://creativecommons.org/licenses/by-nc-sa/4.0/}.

Este texto ha sido preparado con la plantilla \LaTeX{} de TFG (ESI-UCLM) publicada por \href{https://www.uclm.es/profesorado/jsalido}{Jesús Salido} en \href{https://github.com/JesusSalido/TFG_ESI_UCLM}{GitHub} y \href{https://www.overleaf.com/latex/templates/plantilla-de-tfg-escuela-superior-de-informatica-uclm/phjgscmfqtsw}{Overleaf} como parte del curso \href{http://visilab.etsii.uclm.es/?page_id=1468}{\emph{<<\LaTeX{} esencial para preparación de TFG, Tesis y otros documentos académicos>>}} impartido en la Escuela Superior de Informática de la Universidad de Castilla-La Mancha.

% OJO: Si se incluye el escudo de nucleo de ferrita debe incluirse el párrafo siguiente. En caso contrario puede eliminarse.
El escudo basado en el núcleo de ferrita que acompaña la distribución de este documento ha sido realizado por Francisco Moya, David Villa e Ignacio Díez y su inclusión en el documento final debe respetar los derechos de la licencia CC BY-SA 3.0 con la que se distribuye. 

%Para citar este documento puede emplear el registro siguiente:
%
%@Www{salidoTFGgit,
%  author       = {Jesús Salido},
%  title        = {Plantilla guía de TFG para la ESI-UCLM},
%  year         = {2019},
%  editor       = {GitHub},
%  organization = {Universidad de Castilla-La Mancha},
%  url          = {https://github.com/JesusSalido/TFG_ESI_UCLM},
%}

La copia y distribución de esta obra está permitida en todo el mundo, sin regalías y por cualquier medio, siempre que esta nota sea preservada. Se concede permiso para copiar y distribuir traducciones de este libro desde el español original a otro idioma, siempre que la traducción sea aprobada por el autor del libro y tanto el aviso de copyright como esta nota de permiso, sean preservados en todas las copias.}{by-nc-sa}

%---


