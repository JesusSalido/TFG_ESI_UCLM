\chapter{Resultados}
\label{cap:Resultados}

En esta sección se describirá la aplicación del método de trabajo presentado en el capítulo \ref{cap:Metodologia},  mostrando los elementos (modelos, diagramas, especificaciones, etc.) más importantes. 

Este apartado debe explicar cómo el empleo de la metodología permite satisfacer tanto el objetivo principal como los específicos planteados en el TFG así como los requisitos exigidos (según exposición en cap.~\ref{cap:Objetivo}).

\section{Costes, planificación y presupuesto}
Si el TFG consiste en el desarrollo e implementación de un prototipo, la memoria  debe incluir el coste del prototipo considerando tanto el hardware como los recursos humanos necesarios para su desarrollo. Esta explicación se puede incluir en secciones dentro del capítulo de resultados. Si además el TFG contempla la puesta en marcha de la solución diseñada, la planificación y presupuesto ---de la puesta en marcha--- pueden constituir capítulos específicos subsiguientes en la memoria, ya que en sí misma la puesta en marcha constituye un proyecto de ejecución separado del diseño. Es importante no confundir la planificación ---la más habitual--- que se incluye en el capítulo \ref{cap:Metodologia} de metodología aplicada para el desarrollo del proyecto con la mencionada en este apartado, que se refiere a la planificación prevista de puesta en marcha.

Cuando se tiene en cuenta la puesta en marcha de un proyecto de ingeniería, la planificación y presupuesto que se realizan de modo previo a su ejecución son críticos para gestionar los recursos que permitan alcanzar los objetivos de calidad, temporales y económicos previstos para el proyecto. 

Es muy importante que todas las justificaciones aportadas se sustenten no solo en juicios de valor sino en evidencias tangibles como: historiales de actividad, repositorios de código y documentación, porciones de código, trazas de ejecución, capturas de pantalla, demos, etc.
