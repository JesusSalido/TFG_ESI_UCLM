\chapter{Objetivo}
\label{cap:Objetivo}

Introduce y motiva la problemática (i.e.\emph{\ ¿\textsc{cuál} es el problema que se plantea y \textsc{por qué} es interesante su resolución?})

Debe concretar y exponer detalladamente el problema a resolver, el entorno de 
trabajo, la situación y qué se pretende obtener. También puede contemplar las 
limitaciones y condicionantes a considerar para la resolución del problema 
(lenguaje de construcción, equipo físico, equipo lógico de base o de apoyo, 
etc.). Si se considera necesario, esta sección se puede titular 
\emph{Objetivos del TFG e hipótesis de trabajo}. En este caso, se añadirán 
las hipótesis de trabajo que el/la estudiante pretende demostrar con su trabajo.

Una de las tareas más complicadas al proponer un TFG es plantear su 
\textsf{Objetivo}. La dificultad deriva de la falta de consenso respecto de 
lo que se entiende por \emph{objetivo} en un trabajo de esta naturaleza. En 
primer lugar se debe distinguir entre dos tipos de objetivo:

\begin{enumerate}[(A)]
	\item La \emph{finalidad específica} del TFG que se plantea para resolver una problemática concreta aplicando los métodos y herramientas adquiridos durante la formación académica. Por ejemplo, \emph{<<Desarrollo de una aplicación software para gestionar reservas hoteleras \emph{on-line}>>}.
	
	\item El \emph{propósito académico} que la realización de un TFG tiene en la formación de un graduado. Por ejemplo, la \emph{adquisición de competencias específicas de la especialización} cursada.
\end{enumerate}

En el ámbito de la memoria del TFG se tiene que definir el primer tipo de objetivo, mientras que el segundo tipo es el que se añade al elaborar la propuesta de un TFG presentada ante un comité para su aprobación. \emph{Este segundo tipo de objetivo no se debe incluir en la memoria y en todo caso hacerse en la sección de conclusiones finales.}\footnote{En algunas titulaciones es obligatorio que la memoria explique las competencias específicas alcanzadas con la realización del trabajo.}

Un objetivo bien planteado debe estar determinado en términos del \emph{<<producto final>>} esperado que resuelve un problema específico. Por tanto, debería quedar determinado por un sustantivo \emph{concreto} y \emph{medible}. El objetivo planteado puede pertenecer a una de las categorías que se indica a continuación:
\begin{itemize}
	\item \emph{Diseño y desarrollo de <<artefactos>>}
	(habitual en las ingenierías). Por la naturaleza de los programas informáticos (software), los trabajos que implican su diseño suelen contemplar también el desarrollo o implementación de prototipos. Esto es menos frecuente en otras áreas de ingeniería en las que claramente se separa la fase de diseño o realización de un proyecto, frente a la ejecución del mismo (p.~ej.~ingeniería civil, arquitectura, etc.). 
    
	\item \emph{Estudio} que ofrece información novedosa sobre un tema (usual en las ramas de ciencias y humanidades). 
    
	\item \emph{Validación de una 
	hipótesis} de partida (propio de los trabajos 
	científicos y menos habitual en el caso de los TFG).
\end{itemize}

Estas categorías no son excluyentes, de modo que es posible plantear un trabajo cuyo objetivo sea el diseño y desarrollo de un <<artefacto>> y este implique un estudio previo o la validación de alguna hipótesis para guiar el proceso. En este caso y cuando el objetivo sea lo suficientemente amplio puede ser conveniente su descomposición en elementos más simples hablando de \emph{subobjetivos}. Por ejemplo, un programa informático se puede descomponer en módulos o requerir un estudio previo para plantear un nuevo algoritmo que será preciso validar. La descomposición de un objetivo principal en subobjetivos 
u objetivos secundarios debería ser natural (no forzada), bien justificada y 
sólo pertinente en los trabajos de gran amplitud.

Junto con la definición del objetivo del trabajo se puede especificar los \emph{requisitos} que debe satisfacer la solución aportada. Estos requisitos especifican \emph{características} que debe poseer la solución y \emph{restricciones} que acotan su alcance. En el caso de un trabajo cuyo objetivo es el desarrollo de un <<artefacto>> los requisitos pueden ser \emph{funcionales} y \emph{no funcionales}.

Al redactar el objetivo de un TFG se debe evitar confundir los medios con el 
fin. Así es habitual encontrarse con objetivos definidos en términos de las 
\emph{acciones} (verbos) o \emph{tareas} que será preciso 
realizar para llegar al verdadero objetivo. Sin embargo, a la hora de 
planificar el desarrollo del trabajo si es apropiado descomponer todo el 
trabajo en \emph{hitos} y estos en \emph{tareas} para facilitar dicha 
\emph{planificación}.

La categoría del objetivo planteado justifica modificaciones en la organización genérica de la memoria del trabajo. Así en el caso de estudios y validación de hipótesis el apartado de resultados y conclusiones debería incluir los resultados de experimentación y los comentarios de cómo dichos resultados validan o refutan la hipótesis planteada.

