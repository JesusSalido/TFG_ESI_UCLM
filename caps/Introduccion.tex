\chapter{Introducción}
\label{cap:Introduccion}

Este capítulo aborda la motivación del trabajo. Se trata de señalar la necesidad que lo origina, su actualidad y pertinencia. Puede incluir también un estado de la cuestión (o estado del arte) en el que se revisen estudios o desarrollos previos, explicando en qué medida sirven de base al trabajo que se presenta. La estructura del capítulo podría quedar reflejada en las secciones que se indica.

\section{Motivación}
Responde a la pregunta sobre la necesidad o pertinencia del trabajo.

En particular para este documento, se ha tenido presente que durante la realización de la memoria del TFG es muy importante respetar la guía de estilo de la institución~\cite{esi19}. Por tanto, el empleo de plantillas para un sistema de procesamiento de textos (p.~ej., Word o \LaTeX) puede requerir su adaptación cuando la plantilla mencionada no haya sido suministrada por la institución a la que se dirige el trabajo.

Es muy posible que durante la exposición en este capítulo sea preciso enunciar de un modo conciso el objetivo general del trabajo. Sin embargo, este breve enunciado no debe evitar la necesidad de incluir una sección o incluso un capítulo ---como en esta plantilla--- dedicado a explicar en mayor detalle el objetivo general y los secundarios (también denominados específicos) del TFG.

\section{Contexto disciplinar y tecnológico}
También se puede denominar \emph{<<Antecedentes>>} o \emph{<<Estado del arte>>} cuando se trata de comentar trabajos relacionados que han abordado la cuestión u objetivo planteado.

En esta sección se debería introducir el \emph{contexto disciplinar y tecnológico} en el que se desarrolla el trabajo de modo que ayude a entender con facilidad su ámbito y alcance. 

Puesto que un TFG no tiene que ser necesariamente un trabajo con aportes novedosos u originales, solo es necesario la inclusión de \emph{estado del arte} cuando este contribuya a aclarar aspectos clave del TFG o se desee justificar la originalidad del trabajo realizado.

Para redactar un trabajo académico de modo efectivo se recomienda seguir pautas para obtener un resultado final claro y de lectura fácil, como las expuestas en el blog de Leonor Zozaya~\cite{zozaya17} o el apartado de comunicación eficaz del Departamento de Lengua y Estilo de la UOC~\cite{uoc}.

A la hora de redactar el texto se debe poner especial atención para evitar el plagio respetando los derechos de propiedad intelectual~\cite{uc3m21}. En particular, merece gran atención la inclusión de gráficos e imágenes procedentes de Internet que no sean de elaboración propia. En este sentido se sugiere la consulta del manual de la Universidad de Cantabria~\cite{unican18}. Dicho documento explica de modo conciso cómo incluir imágenes en un trabajo académico de modo apropiado.



\section{Estructura del documento}
Este capítulo suele finalizar con una sección en la que se indica la estructura (capítulos) del documento y el contenido de cada una de las partes en que se divide. Veamos a continuación cómo sería esta sección para este documento en concreto.

A lo largo de los capítulos que componen esta guía se muestran ejemplos de elementos de organización del texto en un documento preparado con \LaTeX{}. Todos estos ejemplos se explican en detalle durante el curso de \LaTeX{} esencial para TFG impartido por J.~Salido~\cite{salido10}. Los ejemplos mencionados, así como los recogidos en obras de referencia, se pueden emplear para adaptar este documento a las necesidades particulares  \cite{lamport94,wikibookLaTex10}. Entre las obras de consulta disponibles sobre \LaTeX{} se recomienda el uso de las obras gratuitas en español~\cite{oetiker14,borbon21} y las guías disponibles en la página web de \href{https://es.overleaf.com/learn}{Overleaf} (en inglés).

En esta plantilla de TFG se ha optado por seguir la estructura que debería presentar un TFG en la \mbox{ESI-UCLM}. Esta estructura consta de los capítulos siguientes:

\begin{enumerate}
\item \textbf{Introducción}. Donde se trata la motivación y se justifica la pertinencia del trabajo. Prosigue con el enunciado conciso del propósito material del trabajo y la descripción del contexto disciplinar y técnico de su abordaje.

\item \textbf{Objetivo}. En el que se detalla el alcance del objetivo general y los secundarios que se persiguen con la realización del trabajo.

\item \textbf{Plan de gestión del trabajo}. Describe todos los aspectos relacionados con la estrategia para abordar las distintas fases del trabajo. 

\item \textbf{Desarrollo}. Donde se explica cómo se han llevado a cabo las fases del trabajo cumpliendo el plan previsto.

\item \textbf{Conclusiones}. En el que se realiza una discusión sobre los objetivos alcanzados y la justificación de la aplicación de las competencias adquiridas durante los estudios culminados con el TFG. También puede incluir una explicación sobre los trabajos derivados y futuros, así como una breve valoración personal.

\item \textbf{Bibliografía}. Lista de las referencias bibliográficas citadas en el texto.

\item \textbf{Anexos}. Contenidos auxiliares que complementan del trabajo.
\end{enumerate}









