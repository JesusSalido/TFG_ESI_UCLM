%\input{./Anexos/PortadaETSII} % Por ej.,: Otras portadas
%\includepdf{fichero_portada.pdf} % Incluso como fichero PDF

% -------------------------------------------------------------------------
% PORTADA PRAL. (1)
% -------------------------------------------------------------------------

\portadaPral % Portada pral.

% -------------------------------------------------------------------------
% PORTADA INTERIOR (2)
% -------------------------------------------------------------------------
\portadaInt %Portada interior


% -------------------------
% CRÉDITOS
% -------------------------
\begin{creditos}
El autor puede elegir el tipo de licencia que desee.
Este documento se distribuye con licencia CC BY-NC-SA 4.0. El texto 
completo de la licencia se puede obtener en 
\url{https://creativecommons.org/licenses/by-nc-sa/4.0/}.
La copia y distribución de esta obra está permitida en todo el mundo, sin regalías y por cualquier medio, siempre que esta nota sea preservada. Se concede permiso para copiar y distribuir traducciones de este libro desde el español original a otro idioma, siempre que la traducción sea aprobada por el autor del libro y tanto el aviso de copyright como esta nota de permiso, sean preservados en todas las copias.

\noindent \includegraphics[width=0.15\linewidth]{by-nc-sa}
% Para citar este plantilla empleando bibtex puedes emplear el registro siguiente:
%@www{salidoTFG,
%  author       = {Jesús Salido},
%  title        = {Plantilla guía de TFG para la ESI-UCLM},
%  year         = {2019},
%  editor       = {GitHub},
%  organization = {Universidad de Castilla-La Mancha},
%  url          = {https://github.com/JesusSalido/TFG_ESI_UCLM},
%  doi          = {10.5281/zenodo.4574562}
%}
\end{creditos}


% -------------------------
% CALIFICACIÓN DEL TRIBUNAL  
% -------------------------
\tribunal % Página opcional para calificación 


% -------------------------
% DEDICATORIA (1 pág. máximo) 
% -------------------------
\begin{dedicatoria} % (no confundir con los agradecimientos)
\emph{A mis estudiantes \\ % A alguien muy especial
Por contribuir a hacer de cada día un reto ilusionante}
\end{dedicatoria}

\pagestyle{plain}	% Páginas sólo con numeración inferior al pie

% -------------------------
% RESÚMENES:
% -------------------------
% EDITAR: Resumen en idioma pral. (máx. 1 pág.) 
\begin{resumenPral}[spanish]
En una página como máximo, el resumen explica de modo conciso la problemática que trata de resolver el trabajo \emph{(<<Qué>>)}, la metodología para  abordar su solución (\emph{<<Cómo>>)} y las principales conclusiones del trabajo. En los trabajos cuyo idioma principal sea el inglés, el orden de \textsf{Resumen} y \textsf{Abstract} se invertirá. 

En concreto este documento debe servir como guía para preparar, con \LaTeX, el TFG en la \href{http://webpub.esi.uclm.es/}{Escuela Superior de Informática} (ESI) de la Univ. de Castilla-La Mancha (UCLM), siguiendo la \href{https://pruebasaluuclm.sharepoint.com/sites/esicr/tfg/SitePages/Inicio.aspx}{normativa de aplicación}. Está disponible tanto en \href{https://github.com/JesusSalido/TFG_ESI_UCLM}{GitHub} como \href{https://www.overleaf.com/latex/templates/plantilla-de-tfg-escuela-superior-de-informatica-uclm/phjgscmfqtsw}{Overleaf}. Por tanto, se puede emplear, tanto en un equipo con \LaTeX{} (modo local) instalado, o bien empleando el servicio de edición \href{https://www.overleaf.com/latex/templates/plantilla-de-tfg-escuela-superior-de-informatica-uclm/phjgscmfqtsw}{Overleaf} (modo online).

Este texto se aprovecha para proporcionar información sobre la elaboración de la memoria del TFG con ayuda de \LaTeX{} empleando este documento como plantilla. Por este motivo, sigue una estructura similar a la que se espera encontrar en un TFG, mostrando ejemplos de uso de distintos elementos y comandos de maquetación de textos que se amplía en el anexo~\ref{cap:AnexoA}.

\noindent\emph{IMPORTANTE: Aunque la plantilla se ajusta a las necesidades y reglamentación de la ESI-UCLM, se puede adaptar fácilmente a otras titulaciones, instituciones y otros documentos de carácter académico. Esta plantilla permite la elaboración automática del documento en idioma inglés en cualquier SO (Windows, Linux, Mac OSX, etc.).}
\end{resumenPral}


% EDITAR: Resumen en idioma alternativo (máx. 1 pág.)
%---
\begin{resumenAlt}[english]
\emph{English version for the abstract.}
\end{resumenAlt}


% Ajuste al idioma pral.
\ifbool{ESI@spanish}{\selectlanguage{spanish}}{\selectlanguage{english}}

% -------------------------
% AGRADECIMIENTOS (recomendable máx. 1 pág.)
% -------------------------
\auxchapter{Agradecimientos}
Aunque es un apartado opcional, haremos bueno el refrán \emph{<<es de bien nacidos, ser agradecidos>>} si empleamos este espacio como un medio para agradecer a todos los que, de un modo u otro, han hecho posible que el trabajo realizado \emph{llegue a buen puerto}. Esta sección es ideal para agradecer a directores, profesores, mentores, familiares, compañeros, amigos, etc. 
 
Estos agradecimientos pueden ser tan personales como se desee e incluir anécdotas y chascarrillos, pero recuerda que \emph{no deberían ocupar más de una página}.

\firma % Nombre, lugar y año (automáticamente)

% -------------------------
% -NOTACIÓN: Lista de símbolos con significado especial.
% -------------------------
\auxchapter{Notación y acrónimos}
\section*{Notacion}
Ejemplo de lista con notación (o nomenclatura) empleada en la memoria del TFG. Debe editarse según las necesidades personales.\footnote{Se incluye únicamente con propósito de ilustración, ya que el documento no emplea la notación aquí mostrada.}

\begin{tabular}{r r p{0.8\linewidth}}
$A, B, C, D$	& : & Variables lógicas \\
$f, g, h$		& :	& Funciones lógicas \\
$\cdot$			& : & Producto lógico (AND), a menudo se omitirá como en $A 
B$ en lugar de $A \cdot B$\\
$+$				& : & Suma aritmética o lógica (OR) dependiendo del 
contexto\\
$\oplus$		& : & OR exclusivo (XOR)\\
$\overline{A}$ o ${A}'$	& : & Operador NOT o negación
\end{tabular}

\section*{Lista de acrónimos}
% OJO: Esta lista debería estar ordenada alfabeticamente (hacer de modo manual).
Ejemplo de lista \emph{ordenada alfabéticamente} con los acrónimos empleados en el texto.\footnote{Se pueden omitir aquellos acrónimos que son reconocidos en el contexto académico (p.~ej., PhD), aunque aquí se han incluido a efectos ilustrativos.}

\begin{tabular}{r r p{0.8\linewidth}}
CASE& : &Computer-Aided Software Engineering \\
CTAN& : &Comprenhensive \TeX{} Archive network \\
IDE& : &Integrated Development Environment \\
ECTS& : &European Credit Transfer and Accumulation System \\
OOD& : &Object-Oriented Design \\
PhD& : &Philosophiae Doctor \\
RAD& : &Rapid Application Development \\
SDLC& : &Software Development Life Cycle \\
SSADM& : &Structured Systems Analysis \& Design Method \\
TFE& : &Trabajo Fin de Estudios \\
TFG& : &Trabajo Fin de Grado \\
TFM& : &Trabajo Fin de Máster \\
UML& : &Unified Modeling Language
\end{tabular}

% -------------------------
% ÍNDICES: Si algún índice no existe, se puede eliminar.
% -------------------------
\idxGral
\idxFiguras
\idxTablas
\idxListados
\idxAlgoritmos
%---


