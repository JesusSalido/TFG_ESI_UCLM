%\input{./Anexos/PortadaETSII} % Por ej.,: Otras portadas
%\includepdf{fichero_portada.pdf} % Incluso como fichero PDF

%% -------------------------------------------------------------------------
%% PORTADA PRAL. (1)
%% Puedes incluir directamente la portada realizada por un programa externo.
%% Y directamente modificar el diseño que se proporciona. 
%% NOTA: Para eliminar líneas sin alterar el espaciado original se recomienda emplear \phantom{text}
\begin{titlepage}
    \makeatletter
	\begin{center}
        \pdfbookmark{Portada}{portada}
        \vspace{1cm}
		\includegraphics[width=4.5cm]{\@escudo}\vspace{1cm} 
		
		{\LARGE \textbf{\@instEdu\\[0.5ex]
				\@centroEdu\\[2cm]
				\@titulacion}}\\[0.5cm]
        {\large \textbf{\@especialidad}}\\[1.5cm] 
		{\LARGE \textbf{\@tipoDoc}}\\[1cm]	
		{\LARGE \@tituloPrimera}\\ \smallskip%			
		\ifdefined\@tituloSegunda{\LARGE \@tituloSegunda}\\[3cm]
		\else \phantom{\LARGE	Texto fantasma}\\[3cm]
		\fi
		{\Large \@autor}\vfill%
	\end{center}
	
	\begin{flushright}
		{\Large \ifbool{ESI@spanish}{\@mesTF}{\@monthTF}, \@yearTF}
	\end{flushright}
    \makeatother
\end{titlepage}

% -------------------------------------------------------------------------
% PORTADA INTERIOR (2)
% -------------------------------------------------------------------------
\begin{titlepage}
    \makeatletter
	\begin{center}
        \vspace{1cm}
		\includegraphics[width=4.55cm]{\@escudo}\vspace{1cm}
		
		{\LARGE \textbf{\@instEdu \\[0.5ex]
				\@centroEdu}}\\[0.5cm]
		{\Large \textbf{\@depto}}\\ [1cm]%
		{\large \textbf{\@especialidad}}\\[1.5cm]
		{\LARGE \textbf{\@tipoDoc}}\\[1cm]
		
		{\LARGE \textbf{\@tituloPrimera}}\\ \smallskip%		
		\ifdefined\@tituloSegunda{\LARGE \textbf{\@tituloSegunda}}
		\else \phantom{\LARGE Texto fantasma}
		\fi
	\end{center}
	\vfill%
	\begin{flushleft}
		{\Large Autor: \@autor} \\ \bigskip% (EN: Author)
		{\Large Tutor(a): nombre y apellidos} \\ \bigskip% (EN: Co-Supervisor)
		{\Large Co-tutor(a): nombre y apellidos}
	\end{flushleft}
	\vspace{2cm}%
	\begin{flushright}
		{\Large \ifbool{ESI@spanish}{\@mesTF}{\@monthTF}, \@yearTF}
	\end{flushright}
	\cleardoublepage
    \makeatother
\end{titlepage}
% -------------------------------------------------------------------------
% -------------------------------------------------------------------------

% -------------------------
% CRÉDITOS
% -------------------------
\begin{creditos}
El autor puede elegir el tipo de licencia que desee.
Este documento se distribuye con licencia CC BY-NC-SA 4.0. El texto 
completo de la licencia se puede obtener en 
\url{https://creativecommons.org/licenses/by-nc-sa/4.0/}.
La copia y distribución de esta obra está permitida en todo el mundo, sin regalías y por cualquier medio, siempre que esta nota sea preservada. Se concede permiso para copiar y distribuir traducciones de este libro desde el español original a otro idioma, siempre que la traducción sea aprobada por el autor del libro y tanto el aviso de copyright como esta nota de permiso, sean preservados en todas las copias.

\noindent \includegraphics[width=0.15\linewidth]{by-nc-sa}
% Para citar este plantilla empleando bibtex puedes emplear el registro siguiente:
%@www{salidoTFG,
%  author       = {Jesús Salido},
%  title        = {Plantilla guía de TFG para la ESI-UCLM},
%  year         = {2019},
%  editor       = {GitHub},
%  organization = {Universidad de Castilla-La Mancha},
%  url          = {https://github.com/JesusSalido/TFG_ESI_UCLM},
%  doi          = {10.5281/zenodo.4574562}
%}
\end{creditos}


\tribunal % Página opcional para calificación 

% -------------------------
% DEDICATORIA (1 pág. máximo) comentar si no se desea incluir.
% -------------------------
\begin{dedicatoria} % (no confundir con los agradecimientos)
\emph{A mis estudiantes \\ % A alguien muy especial
Por contribuir a hacer de cada día un reto ilusionante}
\end{dedicatoria}


\pagestyle{plain}	% Páginas sólo con numeración inferior al pie
% -------------------------
% RESÚMENES:
% -------------------------
% EDITAR: Resumen (máx. 1 pág.)
\phantomsection  % Necesario con hyperref
\addcontentsline{toc}{chapter}{Resumen} % Añade al TOC.
\selectlanguage{spanish} % Selección de idioma del resumen.
\makeatletter
\begin{center} %
   {\textsc{TRABAJO FIN DE GRADO - \@centroEdu{}
   (UCLM)}\par} % Tipo de trabajo
   \vspace{1cm} %  
   {\textbf{\Large\@tituloCorto}\par}  % Título del trabajo
   \vspace{0.4cm} %
   {\@autor \\ \@cityTF,{} \@mesTF{} \@yearTF\par} 
   \vspace{0.9cm} %
   {\textbf{\large\textsf{Resumen}}\par} % Título de resumen
\end{center}   
\makeatother %
En una página como máximo, el resumen explica de modo conciso la problemática que trata de resolver el trabajo \emph{(<<Qué>>)}, la metodología para  abordar su solución (\emph{<<Cómo>>)} y las principales conclusiones del trabajo. En los trabajos cuyo idioma principal sea el inglés, el orden de \textsf{Resumen} y \textsf{Abstract} se invertirá. 

En concreto este documento debe servir como guía para preparar, con \LaTeX, el TFG en la \href{http://webpub.esi.uclm.es/}{Escuela Superior de Informática} (ESI) de la Univ. de Castilla-La Mancha (UCLM), siguiendo la \href{https://pruebasaluuclm.sharepoint.com/sites/esicr/tfg/SitePages/Inicio.aspx}{normativa de aplicación}. Está disponible tanto en \href{https://github.com/JesusSalido/TFG_ESI_UCLM}{GitHub} como \href{https://www.overleaf.com/latex/templates/plantilla-de-tfg-escuela-superior-de-informatica-uclm/phjgscmfqtsw}{Overleaf}. Por tanto, se puede emplear, tanto en un equipo con \LaTeX{} (modo local) instalado, o bien empleando el servicio de edición \href{https://www.overleaf.com/latex/templates/plantilla-de-tfg-escuela-superior-de-informatica-uclm/phjgscmfqtsw}{Overleaf} (modo online).

Este texto se aprovecha para proporcionar información sobre la elaboración de la memoria del TFG con ayuda de \LaTeX{} empleando este documento como plantilla. Por este motivo, sigue una estructura similar a la que se espera encontrar en un TFG, mostrando ejemplos de uso de distintos elementos y comandos de maquetación de textos que se amplía en el anexo~\ref{cap:AnexoA}.

\noindent\emph{IMPORTANTE: Aunque la plantilla se ajusta a las necesidades y reglamentación de la ESI-UCLM, se puede adaptar fácilmente a otras titulaciones, instituciones y otros documentos de carácter académico. Esta plantilla permite la elaboración automática del documento en idioma inglés en cualquier SO (Windows, Linux, Mac OSX, etc.).}

\bigskip
\noindent\textbf{Palabras clave}: \keywordslist.

%---
\cleardoublepage % Se incluye para modificar el contador de página antes de añadir 

% EDITAR: Abstract (máx. 1 pág.)
%---
\phantomsection  % Necesario con hyperref
\addcontentsline{toc}{chapter}{Abstract} % Añade al TOC.
\selectlanguage{english} % Selección de idioma del resumen.
\makeatletter
\begin{center} %
   {\textsc{BACHELOR DISSERTATION - \@centroEdu{}
   (UCLM)}\par}
   \vspace{1cm} %  
   {\textbf{\Large Guided template for TFG}\par}
   \vspace{0.4cm} %
   {\@autor \\ \@cityTF,{} \@monthTF{} \@yearTF\par} 
   \vspace{0.9cm} %
   {\textbf{\large\textsf{Abstract}}\par} 
\end{center}   
\makeatother %
\emph{English version for the abstract.}

\bigskip 
\noindent\textbf{Keywords}: \keywordslist.

\ifbool{ESI@spanish}{\selectlanguage{spanish}}{\selectlanguage{english}}
% -------------------------
% AGRADECIMIENTOS (recomendable máx. 1 pág.)
% -------------------------
\auxchapter{Agradecimientos}
Aunque es un apartado opcional, haremos bueno el refrán \emph{<<es de bien nacidos, ser agradecidos>>} si empleamos este espacio como un medio para agradecer a todos los que, de un modo u otro, han hecho posible que el trabajo realizado \emph{llegue a buen puerto}. Esta sección es ideal para agradecer a directores, profesores, mentores, familiares, compañeros, amigos, etc. 
 
Estos agradecimientos pueden ser tan personales como se desee e incluir anécdotas y chascarrillos, pero recuerda que \emph{no deberían ocupar más de una página}.

\firma % Nombre, lugar y año (automáticamente)

% -------------------------
% -NOTACIÓN: Lista de símbolos con significado especial.
% -------------------------
\auxchapter{Notación y acrónimos}
\section*{Notacion}
Ejemplo de lista con notación (o nomenclatura) empleada en la memoria del TFG.\footnote{Se incluye únicamente con propósito de ilustración, ya que el documento no emplea la notación aquí mostrada.}

\begin{tabular}{r r p{0.8\linewidth}}
$A, B, C, D$	& : & Variables lógicas \\
$f, g, h$		& :	& Funciones lógicas \\
$\cdot$			& : & Producto lógico (AND), a menudo se omitirá como en $A 
B$ en lugar de $A \cdot B$\\
$+$				& : & Suma aritmética o lógica (OR) dependiendo del 
contexto\\
$\oplus$		& : & OR exclusivo (XOR)\\
$\overline{A}$ o ${A}'$	& : & Operador NOT o negación
\end{tabular}

\section*{Lista de acrónimos}
% OJO: Esta lista debería estar ordenada alfabeticamente (hacer de modo manual).
Ejemplo de lista \emph{ordenada alfabéticamente} con los acrónimos empleados en el texto.\footnote{Se pueden omitir aquellos acrónimos que son reconocidos en el contexto académico (p.~ej., PhD), aunque aquí se han incluido a efectos ilustrativos.}

\begin{tabular}{r r p{0.8\linewidth}}
CASE& : &Computer-Aided Software Engineering \\
CTAN& : &Comprenhensive \TeX{} Archive network \\
IDE& : &Integrated Development Environment \\
ECTS& : &European Credit Transfer and Accumulation System \\
OOD& : &Object-Oriented Design \\
PhD& : &Philosophiae Doctor \\
RAD& : &Rapid Application Development \\
SDLC& : &Software Development Life Cycle \\
SSADM& : &Structured Systems Analysis \& Design Method \\
TFE& : &Trabajo Fin de Estudios \\
TFG& : &Trabajo Fin de Grado \\
TFM& : &Trabajo Fin de Máster \\
UML& : &Unified Modeling Language
\end{tabular}

% -------------------------
% ÍNDICES: Si algún índice no existe, se puede eliminar.
% -------------------------
\idxGral
\idxFiguras
\idxTablas
\idxListados
\idxAlgoritmos
%---


