% !TeX encoding = UTF-8
%%%%%%%%%%%%%%
% Fichero: uclmTFGesi.tex
% Autor: Jesús Salido Tercero (http://www.uclm.es/profesorado/jsalido)
% Fecha (creación): Febrero 2010 
% Rev. : Febrero 2020
% Descripción: Plantilla para memoria de TFG 
% (Escuela Sup. de Informática, UCLM). Creada para el curso 
% “LaTeX esencial para preparación de TFG, Tesis y otros documentos 
% académicos” (Esc. Sup. Informática-UCLM)
%
%### Compilación 
%
% Esta plantilla ha sido preparada para compilarse con `pdflatex`, `biblatex` 
% (bibliografía con `biber`) y `makeindex` (sólo si se incluye índice 
% temático).
%
% Para su compilación se aconseja utilizar `latexmk` (requiere para su 
% ejecución de un intérprete [`Perl`](http://strawberryperl.com/)):
%
%> \$> latexmk -pdf -silent -synctex=1 --enable-write18 
%
% Para la automatización del trabajo con esta plantilla es recomendable el 
% empleo de IDE dedicados como [TeXstudio](https://www.texstudio.org/).
%
% Una versión revisada de esta plantilla está disponible en overleaf.
% Puede crearse un proyecto propio para escribir un TFG directamente en 
% overleaf, o bien descargarla como un archivo .zip para su utilización en 
% modo local.
%
% Si deseas acceder a la versión de desarrollo puedes encontrarla en GitHub:
%	https://github.com/JesusSalido/TFG_ESI_UCLM
%%%%%%%%%%%%%%



% -------------------------
%
% PREÁMBULO del documento
%
% -------------------------
	
\documentclass[ 		% Clase del documento
	11pt,				% Tamaño de letra
	a4paper,			% Tamaño de papel
	twoside,			% Impresión a doble cara
	openright,			% La apertura de cap. a la dcha.
	final       		% Versión final
]{book}

\usepackage[utf8]{inputenx} % Codificación de entrada (mejora respecto a inputenc)
\usepackage[english,spanish,es-tabla,es-noindentfirst]{babel} % Internacionalización


%--- Geometría de las páginas del documento
\usepackage[			% Márgenes del documento
	top=2.5cm,			% Margen superior
	bottom=2.5cm,		% Margen inferior
	inner=3.5cm,		% Margen al interior
	outer=2cm			% Margen al exterior
]{geometry}


%--- Tipografía
\usepackage{amsmath,amsthm,amssymb}	% OPT.: Mejoras cuando hay matemáticas


%--- Tipografía (Opción 1)
\usepackage[tt=false]{libertine} % Libertine con Old-Style Figures [osf]
\usepackage[libertine]{newtxmath} % Times
%---


%--- Tipografía (Opción 2)
%\usepackage{newpxtext} % Palatino: La opción osf proporciona números en old style.
%\usepackage{newpxmath}	% Palatino
%---


%--- Tipografía (Opción 3)
%\usepackage{fourier} % Utopía
%---


%--- Tipografía (Opción 4)
% EDITAR: Si es preciso cambio de tipo de familia de tipografía por defecto a Sans-Serif
% Aunque es una opción extraña es la preferida en algunos centros docentes (ADE-UCLM).
% Con esta elección es más conveniente una tipografía tipo Helvética/Arial (no Libertine)
%\usepackage{helvet}
%\renewcommand{\familydefault}{\sfdefault}
%---


\usepackage{textcomp,marvosym,pifont} % OPT.: Generación de símbolos 
%especiales
\usepackage{ccicons} % OPT.: Iconos de licencia Creative Commons


\usepackage[T1]{fontenc}% Codificación de salida    
\usepackage{microtype}	% Mejoras de microtipografía en la obtención de PDF (sólo para pdflatex)


%--- Paquete con personalización local para el TFG (ESI-UCLM)
% EDITAR: Si es preciso que la memoria emplee "english" como idioma 
% principal, prefijos de género y ubicación en número de página al pie.
% Con la opción "english" el resto de opciones son irrelevantes.
% Los prefijos de género permiten un tratamiento adecuado en portadas
% automáticas.
%
% Opciones del paquete: (por defecto) Memoria en español y prefijos masculinos.
%	spanish: idioma pral. español (valor por defecto).
%   english: idioma pral. inglés.
%	autora,tutora,cotutora: indica el género de intervinientes (masculino por defecto) 	
%	pageonfooter: Número de página en el pie y centrado como ADE-UCLM (por defecto en cabeceras)
\usepackage{uclmTFGesi}
%\usepackage[pageonfooter]{uclmTFGesi}

% EDITAR: No comentar si es necesario suprimir el sangrado de párrafos.
%\setlength{\parindent}{0pt} % Elimina sangrado de párrafos


%--- Gráficos y tablas
\usepackage{graphicx}	% Inclusión de figuras
\usepackage{subcaption}	% OPT.: Inclusión de subfiguras
% EDITAR: Si es necesario cambiar el path para los directorios de figuras.
\graphicspath{{./figs/}}% Path de búsqueda de ficheros gráficos
\DeclareGraphicsExtensions{.pdf,.png,.jpg} % Precedencia de extensiones
\usepackage{rotating}	% OPT.: Giro de cajas (texto, figuras, tablas) (No 
%DVI)
\usepackage{tabularx,booktabs}	% OPT.: Ajustes para tablas


%--- Personalización de títulos de figuras y tablas
\usepackage[%
	margin=10pt,		% Margen
	font=small,			% Tamaño de tipografía
	labelfont=bf,		% Prefijo-Etiqueta en negrita
	format=hang			%
]{caption}
\captionsetup[table]{skip=5pt} 	% Separación del caption en las tablas
\captionsetup[figure]{skip=5pt} % Separación del caption en las figuras




%--- Bibliografía: Biblatex con biber.
% EDITAR: Si se desea cambiar los estilos de citación y ordenación de la bibliografía
\usepackage[%
	backend=biber,
	% Estilos: numeric, numeric-comp, alphabetic, authoryear, authoryear-comp
	% Otros: apa, chicago, ieee, mla-new, iso-numeric, iso-authoryear
	% Estilos tradicionales BibTeX: trad-plain, trad-unsrt, trad-alpha y trad-abbrv
%	style=apa, % Estilo APA
	style=ieee,
	% Citación: numeric, numeric-comp, numeric-verb, 
	%           authoryear, authoryear-comp,...
	%           otros aplicados con estilo gral.: ieee,apa,aml,chem-acs,iso-numeric,iso-authoryear
	citestyle=numeric-comp,
	sortcites, % Ordenación de citas múltiples cuando son numéricas
	maxbibnames=3, % Máximo número listado de autores en la bibliografía
	minbibnames=1, % Mínimo número de autores cuando se abrevia la lista de autores
	% Descomentar las opciones siguientes para bibliografía multilingüe
	autolang=other, % Requerido para opción multilingüe
	language=auto,   % Requerido para opción multilingüe
	sorting=nyt%
					% Para cambiar criterio de ordenación de las referencias.
 					% =nty (name-title-year), nyt (name-year-title), nyvt (name-year-volume-title), 
					% =anyt (alphabetic-name-year-title), anyvt (alphabetic-name-year-volume-title), 
					% =ynt (year-name-title), ydnt (yeardescendent-name-title), 
					% =none (por orden de citación, como en ETSII-UCLM).			
]{biblatex}


% Línea añadida para eliminar el idioma de la fuente bibliográfica.
\AtEveryBibitem{\clearfield{note} \clearlist{language}}
% EDITAR: Si es preciso cambiar el fichero de bibliografía. 
\addbibresource{biblioTFG.bib} 	% Fichero de bibliografía.
%---

\usepackage{makeidx} % OPT.: Indice temático
\makeindex           % OPT.: Procesamiento de índice temático


%--- OPT.: Paquete para incluir menús, paths y teclas de modo "elegante"
\usepackage[os=win,hyperrefcolorlinks]{menukeys} 
% OJO: Este paquete presenta algunas incompatibilidades, debe cargarse el 
% último y exige la desactivación de la opción colorlinks y la inclusión en 
% este paquete de la opción hyperrefcolorlinks.
% Este paquete presenta alguna incompatibilidades por lo que cambiarlo de 
% ubicación puede generar algún error (manejar con cuidado).

% Estas definiciones permiten cambiar el estilo de los elementos. Si se desean otros estilos o su configuación es preciso recurrir a la documentación del paquete (no lo recomiendo).
\renewmenumacro{\menu}[>]{menus} % OPT.: default: menus
\renewmenumacro{\directory}[/]{pathswithblackfolder} % OPT.: default: paths
\renewmenumacro{\keys}[+]{shadowedroundedkeys} % OPT.: default: roundedkeys
%---

% -------------------------
% -------------------------
% -------------------------



% -------------------------
% -------------------------
% -------------------------
% DATOS DEL DOCUMENTO 
% Definición de variables empleadas en el documento por lo que no son
% traducidos. Cuando algún campo puede tener varias líneas aparecen dos
% campos señalados como <campo>Primera y <campo>Segunda. Si no se desea 
% emplear los campos opcionales (OPT.) estos deben comentarse.
% -------------------------
% EDITAR: Datos del documento. 
% NOTA: Los elementos opcionales pueden eliminarse o comentarse.
\tituloPrimera{Plantilla guía de TFG para la ESI-UCLM} % 1ª Línea
\tituloSegunda{Curso de {\LaTeX{}} esencial} % OPT.: Para títulos largos.
%%existe).
\titulo{Plantilla guía de TFG} % Título corto (mostrado en pág. de créditos)
\autor{Jesús Salido Tercero}
\email{jesus.salido@uclm.es}					
\tutor{<tutor(a) (nombre apellidos)>} % Sólo nombre, el prefijo añadido automát.
\cotutor{<co-tutor(a) (nombre apellidos)>}	% OPT.: Cotutor(a)
\instEdu{UNIVERSIDAD DE CASTILLA-LA MANCHA}

% Fichero con escudo de la institución
% Logo de la ESI que prefieras (la normativa no especifica obligatoriedad).
%\escudo{esi} 					% Logo ESI (gris uniforme)							
%\escudo{esi_black} 			% Logo ESI (negro)						
%\escudo{esi_color}				% Logo ESI (dos tintas)						
\escudo{IngInformatica_color}	% Nucleo de ferrita	(color)
%\escudo{IngInformatica_bw} 	% Nucleo de ferrita	(en escala de grises)					
%\escudo{etsii_color} 			% Escudo ETSII

\centroEdu{ESCUELA SUPERIOR DE INFORMÁTICA}
\deptoEduPrimera{<Primera línea Depto. Director>} % 1ª Línea(EN: Department of ...)
\deptoEduSegunda{<Segunda línea Depto. Director>} % OPT.: Para nombres largos.
\titulacion{GRADO EN INGENIERÍA INFORMÁTICA} % (EN: BACHELOR IN COMPUTING ENGINEERING)
\especialidad{<Tecnología Específica>} % OPT.: Especialidad - Intensificación
% (EN: Specialization in ...)
\tipoDoc{TRABAJO FIN DE GRADO} % (EN: BACHELOR DISSERTATION)

% Si las fechas se desean en inglés hay que ponerla explícita.
\fechaDef{julio, 2020} 			% Fecha de defensa
\mesDef{julio}        			% Mes de defensa
\yearDef{2020}        			% Año de defensa
\lugarDef{Ciudad Real}			% Lugar de defensa


% --- Metadatos (propiedades) para el documento PDF
\hypersetup{% OPT.
% EDITAR: Valores para el PDF.	
	pdftitle={Plantilla guía para TFG en la UCLM}, % Título
	pdfauthor={Jesús Salido (ESI-UCLM)}, % Autor
	pdfsubject={TFG},  % Tema
	pdftoolbar=true, % Muestra la toolbar de Acrobat
	pdfmenubar=true	 % Muestra la menubar de Acrobat
}
% -------------------------
% -------------------------
% -------------------------










% -------------------------
% -------------------------
% -------------------------
% -------------------------
%
% CUERPO DEL DOCUMENTO
%
% -------------------------
\begin{document}
\frontmatter
% Cambia la numeración de páginas a números romanos y las secciones no están 
%numeradas aunque si aparecen en el índice de contenidos.
\pagestyle{empty}  % Páginas sin cabecera ni pies

% -------------------------
%
% PORTADAS: 
% Los comandos \portadaTFG y \portadilla generan dos portadas con LaTeX 
% teniendo en cuenta los datos sobre el documento aportados en el preámbulo.
% EDITAR: Si se desea incluir una portada generada externamente en PDF se emplea la
% versión del comando con estrella indicando el fichero.
%
% -------------------------
\portadaTFG		% Portada pral.
%\portadaTFG*{./portadas/portadaADE} % Portada generada externamente en PDF
%\portadaTFG*{./portadas/portadaETSII_IE} % Portada generada externamente en PDF
%\portadaTFG*{./portadas/portadaETSII_TFM} % Portada generada externamente en PDF
%---
% EDITAR: Si es preciso incluir portada generada externamente en PDF.
\portadillaTFG	% Portada interior (con tutor(a) y co-tutor(a) si existe).
% También en versión con estrella para indicar fichero PDF
% Comentar si no se desea incluir
%---

% -------------------------
%
% OPT.: CRÉDITOS (aunque no es obligatorio es recomendable).
%
% -------------------------
% -------------------------
%
% CRÉDITOS
%
% -------------------------
% EDITAR (opcional): Licencia (si se desea modificar).
% Este comando permite una gran flexibilidad y la ventaja de no depender de paquetes externos.
% Esta es una página reservada para señalar información relativa a los derechos de autor y la licencia de distribución y uso del documento. Esta página debería ser aprovechada también para informar de cualquier tipo de cesión de los derechos anteriormente citados. El autor del TFG debe tener presente que el incumplimiento de la legislación vigente en materia de protección de la propiedad intelectual es de su exclusiva responsabilidad independientemente de la cesión de derechos que este haya convenido para su obra ya que no son objeto de cesión aquellos derechos de los que no se es poseedor.

\creditos{Este documento se distribuye con licencia Creative Commons Atribución Compartir Igual 4.0. El texto completo de la licencia puede obtenerse en \url{https://creativecommons.org/licenses/by-sa/4.0/}. 
% El escudo de Informática basado en el núcleo de ferrita que acompaña la distribución de esta plantilla ha sido realizado por P.~Moya, D.~Villa e I. Díez, su inclusión debe respetar los derechos de autor y las licencias a las que se vea sometido. 

La copia y distribución de esta obra está permitida en todo el mundo, sin regalías y por cualquier medio, siempre que esta nota sea preservada. Se concede permiso para copiar y distribuir traducciones de este libro desde el español original a otro idioma, siempre que la traducción sea aprobada por el autor del libro y tanto el aviso de copyright como esta nota de permiso, sean preservados en todas las copias.}{cclicense}
%---



%---

%---
% EDITAR: Si es preciso incluir portada generada externamente en PDF.
\tribunalTFG % Página para calificaciones del tribunal
% También en versión con estrella para indicar fichero PDF
%---

% -------------------------
%
% OPT.: DEDICATORIA (1 pág. máximo) comentar si no se desea incluir.
% Aunque opcional, no se debería perder la oportunidad de poder 
% dedicar el trabajo a alguien MUY especial.
% EDITAR: Dedicatoria.
\dedicado{A mis estudiantes \\ % A alguien muy especial
Por contribuir a hacer de cada día un reto ilusionante} % Como mucho dos líneas (no confundir con los agradecimientos).
% -------------------------

% -------------------------
%
% RESUMEN:
%
% -------------------------
%--- Ajustes del documento.
\pagestyle{plain}	% Páginas sólo con numeración inferior al pie

% -------------------------
%
% RESUMEN:
% 
%

% EDITAR: Resumen (máx. 1 pág.)
%\cleardoublepage % Se incluye para modificar el contador de página antes de añadir bookmark
\phantomsection  % Necesario con hyperref
\addcontentsline{toc}{chapter}{Resumen} % Añade al TOC.
\selectlanguage{spanish} % Selección de idioma del resumen.
\makeatletter
\begin{center} %
   {\textsc{TRABAJO FIN DE GRADO - ESCUELA SUP. DE INFORMÁTICA (UCLM)}\par} % Tipo de trabajo
   \vspace{1cm} %  
   {\textbf{\Large\@tituloCorto}\par}  % Título del trabajo
   \vspace{0.4cm} %
   {\@autor \\ \@cityTF,{} \@mesTF{} \@yearTF\par} 
   \vspace{0.9cm} %
   {\textbf{\large\textsf{Resumen}}\par} % Título de resumen
\end{center}   
\makeatother %
En una página como máximo, el resumen explicará de modo breve la problemática que trata de resolver el trabajo \emph{(<<Qué>>)}, la metodología para  abordar su solución (\emph{<<Cómo>>)} y los resultados obtenidos. En los trabajos cuyo idioma principal sea el inglés, el orden de \textsf{Resumen} y \textsf{Abstract} se invertirá. Recuerda que el resumen sirve como una descripción concisa de tu trabajo, por lo que debe proporcionar suficiente información para que las personas que lo lean comprendan su propósito, los métodos y los resultados obtenidos.

En concreto este documento debe servir como guía para preparar, con LaTeX, el TFG en la \href{http://webpub.esi.uclm.es/}{Escuela Superior de Informática} (ESI) de la Univ. de Castilla-La Mancha (UCLM) siguiendo la \href{https://pruebasaluuclm.sharepoint.com/sites/esicr/tfg/SitePages/Inicio.aspx}{normativa de aplicación}. Está disponible en \href{https://github.com/JesusSalido/TFG_ESI_UCLM}{GitHub} y \href{https://www.overleaf.com/latex/templates/plantilla-de-tfg-escuela-superior-de-informatica-uclm/phjgscmfqtsw}{Overleaf}. Por tanto, se puede emplear tanto en modo local en un equipo con LaTeX instalado (\href{https://miktex.org/}{MiKTeX}, \href{https://www.tug.org/texlive/}{TeX Live}, etc.), o bien en línea empleando el servicio de edición \href{https://www.overleaf.com/latex/templates/plantilla-de-tfg-escuela-superior-de-informatica-uclm/phjgscmfqtsw}{Overleaf}.

Este documento se aprovecha para proporcionar información sobre la elaboración de la memoria del TFG con ayuda de \LaTeX{} empleando este documento como plantilla. Por este motivo, el documento sigue una estructura similar de secciones a la que debe presentar un TFG y muestra ejemplos de uso de distintos elementos y comandos de maquetación del documento.

\emph{Aunque la plantilla se ajusta a las necesidades y reglamentación de la ESI-UCLM, su adaptación es sencilla a otras titulaciones, instituciones y otros documentos de carácter académico. Esta plantilla permite la elaboración automática del documento en idioma inglés en cualquier SO (Windows, Linux, Mac OSX, etc.).}


%---
\cleardoublepage % Se incluye para modificar el contador de página antes de añadir 





% EDITAR: Abstract (máx. 1 pág.)
%---
\phantomsection  % Necesario con hyperref
\addcontentsline{toc}{chapter}{Abstract} % Añade al TOC.
\selectlanguage{english} % Selección de idioma del resumen.
\makeatletter
\begin{center} %
   {\textsc{BACHELOR DISSERTATION - ESCUELA SUP. DE INFORMÁTICA (UCLM)}\par}
   \vspace{1cm} %  
   {\textbf{\Large Guided template for TFG}\par}
   \vspace{0.4cm} %
   {\@autor \\ \@cityTF,{} \@monthTF{} \@yearTF\par} 
   \vspace{0.9cm} %
   {\textbf{\large\textsf{Abstract}}\par} 
\end{center}   
\makeatother %
\emph{English version for the abstract.}
\cleardoublepage % Se incluye para modificar el contador de página antes de añadir 


%---

% -------------------------
%
% AGRADECIMIENTOS (recomendable máx. 1 pág.)
%
% -------------------------
\ifspanish
	\selectlanguage{spanish}
\else
	\selectlanguage{english}
\fi

% -------------------------
%
% AGRADECIMIENTOS (recomendable máx. 1 pág.)
%
% -------------------------
\phantomsection % Necesario con hyperref
\chapter*{Agradecimientos} % Opción con * para que no aparezca en TOC ni numerada
\addcontentsline{toc}{chapter}{Agradecimientos} % Añade al TOC.

% EDITAR: Al gusto
 Aunque es un apartado opcional, haremos bueno el refrán \emph{<<es de bien nacidos, ser agradecidos>>} si empleamos este espacio como un medio para agradecer a todos los que, de un modo u otro, han hecho posible que el trabajo realizado \emph{llegue a buen puerto}. Esta sección es ideal para agradecer a directores, profesores, mentores, familiares, compañeros, amigos, etc. 
 
 Estos agradecimientos pueden ser tan personales como se desee e incluir anécdotas y chascarrillos, pero \emph{nunca deberían ocupar más de una página}.

\makeatletter		
\begin{flushright}
	\vspace{1,5cm}
	\textit{\@autor}\\
	\@lugarDefensa, \@yearDefensa
\end{flushright}
\makeatother % Agradecimientos etc.
%---

% -------------------------
%
% OPT. NOTACIÓN: Lista de símbolos con significado especial.
%
% -------------------------
% -------------------------
%
% -NOTACIÓN: Lista de símbolos con significado especial.
%
% -------------------------
\cleardoublepage
\phantomsection % Necesario con hyperref

% El método mostrado en este fichero es un modo rápido de incluir nomeclatura y listade acrónimos. En trabajos donde se precise un trabajo más depurado e intensivo puede recurrirse a los paquetes:
%   - nomencl
%   - acronym

\chapter*{Notación y acrónimos} % Opción con * para que no aparezca en TOC ni numerada
\addcontentsline{toc}{chapter}{Notación y acrónimos} % Añade al TOC.

\section*{Notacion}
Ejemplo de lista con notación (o nomenclatura) empleada en la memoria del TFG.\footnote{Se incluye únicamente con propósito de ilustración, ya que el documento no emplea la notación aquí mostrada.}

\begin{tabular}{r r p{0.8\linewidth}}
$A, B, C, D$	& : & Variables lógicas \\
$f, g, h$		& :	& Funciones lógicas \\
$\cdot$			& : & Producto lógico (AND), a menudo se omitirá como en $A 
B$ en lugar de $A \cdot B$\\
$+$				& : & Suma aritmética o lógica (OR) dependiendo del 
contexto\\
$\oplus$		& : & OR exclusivo (XOR)\\
$\overline{A}$ o ${A}'$	& : & Operador NOT o negación
\end{tabular}

\section*{Lista de acrónimos}
% OJO: Esta lista debería estar ordenada alfabeticamente (hacer de modo manual).
Ejemplo de lista \emph{ordenada alfabéticamente} con los acrónimos empleados en el texto.\footnote{Se pueden omitir aquellos acrónimos que son reconocidos en el contexto como académico (p.~ej., PhD), aunque aquí se han incluido a efectos ilustrativos.}

\begin{tabular}{r r p{0.8\linewidth}}
CASE& : &Computer-Aided Software Engineering \\
CTAN& : &Comprenhensive \TeX{} Archive network \\
IDE& : &Integrated Development Environment \\
ECTS& : &European Credit Transfer and Accumulation System \\
OOD& : &Object-Oriented Design \\
PhD& : &Philosophiae Doctor \\
RAD& : &Rapid Application Development \\
SDLC& : &Software Development Life Cycle \\
SSADM& : &Structured Systems Analysis \& Design Method \\
TFE& : &Trabajo Fin de Estudios \\
TFG& : &Trabajo Fin de Grado \\
TFM& : &Trabajo Fin de Máster \\
UML& : &Unified Modeling Language
\end{tabular} % Notación empleada.
%---

% -------------------------
%
% ÍNDICES
%
% -------------------------
% -------------------------
%
% ÍNDICES: 
% EDITAR: Si alguno de los índices no existe, su inclusión se puede comentar.
%
% -------------------------
\setindexnames % Ajusta nombres (sólo en español).
\pagestyle{fancy} % Estilo de página ajustado por fancyhdr

%--- Índice general
\cleardoublepage
\phantomsection % OJO: Necesario con hyperref
\pdfbookmark[0]{Índice general}{idx_toc}% idx_toc.0 % Bookmark en PDF
\tableofcontents  % Índice general
% Todos los listados se han incluido en el índice gral. de contenidos. De 
%modo automático también quedan añadidos a los bookmarks del PDF. Si se 
%desean 
%eliminiar del TOC se pueden comentar el comando \addcontensline.
%---

%--- Índice de figuras
\cleardoublepage
\phantomsection % OJO: Necesario con hyperref
\addcontentsline{toc}{chapter}{\listfigurename} % Añade la lista de figuras al TOC (también a bookmarks en PDF)
%\pdfbookmark[0]{\listfigurename}{idx_lof}% idx_lof.0 % Bookmark en PDF
\listoffigures    % Índice de figuras (opcional)
%---

%--- Índice de tablas
\cleardoublepage
\phantomsection % OJO: Necesario con hyperref
\addcontentsline{toc}{chapter}{\listtablename} % Añade la lista de tablas al TOC (también a bookmarks en PDF)
%\pdfbookmark[0]{\listtablename}{idx_lot}% idx_lot.0 % Bookmark en PDF
\listoftables % Índice de tablas (opcional)
%---

%--- Índice de listados
% EDITAR: Comentar todo el bloque para no incluir.
\cleardoublepage
\phantomsection % OJO: Necesario con hyperref
\addcontentsline{toc}{chapter}{\lstlistlistingname} % Añade la lista de listados al TOC (también a bookmarks en PDF)
%\pdfbookmark[0]{\lstlistlistingname}{idx_lol}% idx_lol.0 % Bookmark en PDF
\lstlistoflistings % Índice de listados creados con listings (opcional)
%---

%--- Índice de algoritmos
% EDITAR: Comentar todo el bloque para no incluir.
\cleardoublepage
\phantomsection % OJO: Necesario con hyperref
\addcontentsline{toc}{chapter}{\listalgorithmcfname} % Añade la lista de algoritmos al TOC (también a bookmarks en PDF)
%\pdfbookmark[0]{\listalgorithmcfname}{idx_loa}% idx_loa.0 % Bookmark en PDF
\listofalgorithms % Índice de algoritmos creados con algortihm2e
%---

 % Índice de contenido, figuras, tablas, listados, etc.
%---

%--- MAINMATTER
% Capítulos del documento
% Salva en un contador interno el nº de páginas actual
% Debe ir antes de \mainmatter (antes de que se reinicie el cnt page)
\savepagecnt
\mainmatter
% Justo antes del primer capítulo del libro. Activa la numeración con números arábigos y reinicia el contador de páginas.

% Ajusta valor de cabeceras y pies a comienzo de capítulo
\ifpageonfooter
\else
	\cleanhdfirst
\fi

% Reajuste del número de página consecutivo para no reiniciar paginación en Cáp. 1
%\contpagination % Comentado para reiniciar paginación (pag. 1)

% -------------------------
%
% CAPÍTULOS: Un fichero por capítulo.
%
% -------------------------
\chapter{Introducción}
\label{cap:Introduccion}

Este capítulo aborda la motivación del trabajo. Se trata de señalar la necesidad de la que surge, su actualidad y pertinencia. Puede incluir también un estado de la cuestión en la que se revisen estudios o desarrollos previos y en qué medida sirven de base al trabajo que se presenta.

A continuación se muestran algunos ejemplos para la inclusión de elementos en el documento.

% ------------------------------------------------------------------------------
% Ejemplos para la plantilla
% ------------------------------------------------------------------------------
\section{Ejemplos de listas}
\label{sec:ejListas}
\index{ejemplos} % Véase cómo se incluyen entradas en el índice alfabético
A continuación se van a añadir algunos ejemplos que pueden emplearse al redactar la memoria.

\index{ejemplos!listas} % Para el índice
\noindent Ejemplo de lista con \emph{bullet} especial. 
% Ejemplo: Lista con bullets especiales
% ============
\begin{itemize}
	\item[*] peras
	\item manzanas
	\item[\ding{170}] naranjas
\end{itemize}

\noindent Ejemplo de lista compacta (también se puede emplear el entorno para enumeraciones \emph{compactenum})
% Ejemplo: Lista con balas
% ============
\begin{compactitem}
	\item peras
	\item manzanas
	\item naranjas
\end{compactitem}


\noindent Ejemplo de lista en varias columnas.
% Ejemplo: Listas en varias columnas
% ============
\begin{multicols}{2} % El parámetro es el número de columnas de la lista
	\begin{compactenum}
		\item peras
		\item manzanas
		\item naranjas
		\item patatas
		\item calabazas
		\item fresas
	\end{compactenum}
\end{multicols}


\newpage


\section{Ejemplos de tablas}
\label{sec:ejTablas}
\index{ejemplos!tablas}
A continuación se incluyen algunos ejemplos de tablas hechas con \LaTeX{} y paquetes dedicados.

% Ejemplo: Tabla con macro \cline
% ==========
\begin{table}[htb]%
	\centering
	\caption{Ejemplo de uso de la macro \texttt{cline}}
	\label{tab:cline}
	\begin{tabular}[t]{|r|l|}
		\hline
		7C0 & hexadecimal \\[1cm] % Ejemplo de separación fijada entre líneas
		3700 & octal \\ \cline{2-2}
		11111000000 & binario \\
		\hline \hline
		1984 & decimal \\
		\hline
	\end{tabular}
\end{table}


\noindent Ejemplo de tabla en la que se controla el ancho de la celda.

% Ejemplo: Ejemplo de tabla con control de la anchura de celda.
% ==========
\begin{table}[htb]%
	\centering
	\caption{Ejemplo de tabla con especificación de anchura de columna}
	\label{tab:anchura}
	\begin{tabular}{ | l | l | l | p{5cm} |}
		\hline
		Día & Temp Mín (\textdegree C) & Temp Máx (\textdegree C) & Previsión \\ \hline
		Lunes & 11 & 22 & Día claro y muy soleado. Sin embargo, la brisa de la tarde puede hacer que las temperaturas desciendan \\ \hline
		Martes & 9 & 19 & Nuboso con chubascos en muchas regiones. En Cataluña claro con posibilidad de bancos nubosos al norte de la región \\ \hline
		Miércoles & 10 & 21 & La lluvía continuará por la mañana pero las condiciones climáticas mejorarán considerablemente por la tarde\\
		\hline
	\end{tabular}
\end{table}

\clearpage


\section{Ejemplos de figuras}
\label{sec:ejFiguras}
\index{ejemplos!figuras}

En esta sección se añaden ejemplos de muestra para la inclusión de figuras simples y subfiguras.

% Ejemplo: Ejemplo de inclusión de figura
% ============
\begin{figure}[htb]
	\centering
	\includegraphics[width=0.4\linewidth]{escudoInf}
	\caption[Ejemplo de figura]{Figura vectorial del escudo de la ESI}
	\label{fig:ejFigure}
\end{figure}


\noindent Ejemplo de figuras compuestas por subfiguras.

% Ejemplo: Ejemplo de inclusión de subfiguras
% ============
\begin{figure}[htb]
	\centering
	\subfigure[Gráfico vectorial PDF]{
		\includegraphics[width=0.4\linewidth]{escudoInf}
		\label{fig:escudoColor}
	} 
	\subfigure[Gráfico png]{
		\includegraphics[width=0.4\linewidth]{escudoInfBW}
		\label{fig:escudoBW}
	}
	\caption[Ejemplo de subfiguras]{Ejemplo de inclusión de subfiguras en un mismo entorno}
	\label{fig:ejSubfigures}
\end{figure}


\clearpage


\section{Ejemplos de listados}
\label{sec:ejListados}
\index{ejemplos!listados}

Ejemplos más representativos de inclusión de porciones de código fuente.

% Ejemplo: Listado Java
% ============
\begin{lstlisting}[language=Java,float=ht,caption={[Código fuente en Java]Ejemplo de código fuente en lenguaje Java},label=lst:java]
// @author www.javadb.com
public class Main {    
// Este método convierte un String a
// un vector de bytes

public void convertStringToByteArray() {

String stringToConvert = "This String is 15";      
byte[] theByteArray = stringToConvert.getBytes();        
System.out.println(theByteArray.length);        
}

// argumentos de línea de comandos 
public static void main(String[] args) {
new Main().convertStringToByteArray();
}
}
\end{lstlisting}



\noindent Otro ejemplo.

\begin{lstlisting}[style=C-ruled,float=ht,caption={Ejemplo de código C},label=lst:codC]
// Este código se ha incluido tal cual está 
// en el fichero \LaTeX{}
#include <stdio.h>
int main(int argc, char* argv[]) {
puts("¡Hola mundo!");
}
\end{lstlisting}


\noindent Ejemplo de entrada por consola.

\begin{lstlisting}[style=consola, numbers=none]
$ gcc -o Hola HolaMundo.c
\end{lstlisting}


\subsection{Algoritmos con el paquete \texttt{algorithm2e}}
Como ya se ha comentado en los textos científicos relacionados con las TIC\footnote{Por supuesto en un TFG o tesis de una Escuela de Informática.} (Tecnologías de la Información y Comunicaciones) suelen aparecer porciones de código en los que se explica alguna función o característica relevante del trabajo que se expone. Muchas veces lo que se quiere ilustrar es un algoritmo o método en que se ha resuelto un problema abstrayéndose del lenguaje de programación concreto en que se realiza la implementación. El paquete \texttt{algorithm2e}\footnote{\url{https://osl.ugr.es/CTAN/macros/latex/contrib/algorithm2e/doc/algorithm2e.pdf}} proporciona un entorno \texttt{algorithm} para la impresión apropiada de algoritmos tratándolos como objetos flotantes y con muchas flexibilidad de personalización. En el algoritmo \ref{alg:como} se muestra cómo puede emplearse dicho paquete. En este curso no se explican las posibilidades del paquete más en profundidad ya que excede el propósito del curso. A todos los interesados se les remite a la documentación del mismo.


% Ejemplo:
% ============
\IncMargin{1em}
\begin{algorithm}
\SetKwInOut{Input}{Datos}\SetKwInOut{Output}{Resultado}
\LinesNumbered
\SetAlgoLined

\Input{este texto} 
%\KwIn{este texto}
\Output{como escribir algoritmos con \LaTeX2e}
%\KwOut{como escribir algoritmos con \LaTeX2e}

inicialización\;
\While{no es el fin del documento}{
	leer actual\;
	\eIf{comprendido}{
		ir a la siguiente sección\;
		la sección actual es esta\;
	}{
		ir al principio de la sección actual\;
	}
}

% Aunque el captión aparece abajo siempre se pone arriba como en tablas y listados
\caption{Cómo escribir algoritmos}\label{alg:como}
\end{algorithm}\DecMargin{1em}

\newpage

\section{Menús, paths y teclas con el paquete \texttt{menukeys}}
Cada vez es más usual que los trabajos en ingeniería exijan el uso de software. Para poder especificar de modo elegante el uso menús, pulsación de teclas y directorios se recomienda el uso del paquete \texttt{menukeys}.\footnote{\url{https://osl.ugr.es/CTAN/macros/latex/contrib/menukeys/menukeys.pdf}} Este paquete nos permite especificar el acceso a un menú, por ejemplo:

\menu{Herramientas > Órdenes > PDFLaTeX}\\

\noindent También un conjunto de teclas. Por ejemplo:
\keys{\ctrl + \shift + T}\\

\noindent O un directorio:
\directory{C:/user/LaTeX/Ejemplos}\\

\noindent Aunque este paquete permite muchas opciones de configuración de los estilos aplicados, no es necesario hacerlo para obtener unos resultados muy elegantes.

\chapter{Objetivo}
\label{cap:Objetivo}

Introduce y motiva la problemática (i.e.\emph{\ ¿\textsc{cuál} es el problema que se plantea y \textsc{por qué} es interesante su resolución?})

Debe concretar y exponer detalladamente el problema a resolver, el entorno de 
trabajo, la situación y qué se pretende obtener. También puede contemplar las 
limitaciones y condicionantes a considerar para la resolución del problema 
(lenguaje de construcción, equipo físico, equipo lógico de base o de apoyo, 
etc.). Si se considera necesario, esta sección se puede titular 
\emph{Objetivos del TFG e hipótesis de trabajo}. En este caso, se añadirán 
las hipótesis de trabajo que el/la estudiante pretende demostrar con su trabajo.

Una de las tareas más complicadas al proponer un TFG es plantear su \textsf{Objetivo}. La dificultad deriva de la falta de consenso respecto de lo que se entiende por \emph{objetivo} en un trabajo de esta naturaleza. En primer lugar, se debe distinguir entre dos tipos de objetivo:


\begin{enumerate}[(A)]
	\item La \emph{finalidad específica} del TFG que se plantea para resolver una problemática concreta aplicando los métodos y herramientas adquiridos durante la formación académica. Por ejemplo, \emph{<<Desarrollo de una aplicación software para gestionar reservas hoteleras \emph{on-line}>>}.
	
	\item El \emph{propósito académico} que la realización de un TFG tiene en la formación de un graduado. Por ejemplo, la \emph{adquisición de competencias específicas de la especialización} cursada.
\end{enumerate}

En el ámbito de la memoria del TFG se tiene que definir el primer tipo de objetivo, mientras que el segundo tipo es el que se añade al elaborar la propuesta de un TFG presentada ante un comité para su aprobación. \emph{Este segundo tipo de objetivo no se debe incluir en la memoria y en todo caso hacerse en la sección de conclusiones finales.}\footnote{En algunas titulaciones es obligatorio que la memoria explique las competencias específicas alcanzadas con la realización del trabajo.}

Un objetivo bien planteado debe estar determinado en términos del \emph{<<producto final>>} esperado que resuelve un problema específico. Por tanto, debería quedar determinado por un sustantivo \emph{concreto} y \emph{medible}. El objetivo planteado puede pertenecer a una de las categorías que se indica a continuación:
\begin{itemize}
	\item \emph{Diseño y desarrollo de <<artefactos>>}
	(habitual en las ingenierías). Por la naturaleza de los programas informáticos (software), los trabajos que implican su diseño suelen contemplar también el desarrollo o implementación de prototipos. Esto es menos frecuente en otras áreas de ingeniería en las que claramente se separa la fase de diseño o realización de un proyecto, frente a la ejecución del mismo (p.~ej.~ingeniería civil, arquitectura, etc.). 
    
	\item \emph{Estudio} que ofrece información novedosa sobre un tema (usual en las ramas de ciencias y humanidades). 
    
	\item \emph{Validación de una 
	hipótesis} de partida (propio de los trabajos 
	científicos y menos habitual en el caso de los TFG).
\end{itemize}

Estas categorías no son excluyentes, de modo que es posible plantear un trabajo cuyo objetivo sea el diseño y desarrollo de un <<artefacto>> y este implique un estudio previo o la validación de alguna hipótesis para guiar el proceso. En este caso y cuando el objetivo sea lo suficientemente amplio puede ser conveniente su descomposición en elementos más simples hablando de \emph{subobjetivos}. Por ejemplo, un programa informático se puede descomponer en módulos o requerir un estudio previo para plantear un nuevo algoritmo que será preciso validar. La descomposición de un objetivo principal en subobjetivos 
u objetivos secundarios debería ser natural (no forzada), bien justificada y 
sólo pertinente en los trabajos de gran amplitud.

Junto con la definición del objetivo del trabajo se puede especificar los \emph{requisitos} que debe satisfacer la solución aportada. Estos requisitos especifican \emph{características} que debe poseer la solución y \emph{restricciones} que acotan su alcance. En el caso de un trabajo cuyo objetivo es el desarrollo de un <<artefacto>> los requisitos pueden ser \emph{funcionales} y \emph{no funcionales}.

Al redactar el objetivo de un TFG se debe evitar confundir los medios con el 
fin. Así es habitual encontrarse con objetivos definidos en términos de las 
\emph{acciones} (verbos) o \emph{tareas} que será preciso 
realizar para llegar al verdadero objetivo. Sin embargo, a la hora de 
planificar el desarrollo del trabajo si es apropiado descomponer todo el 
trabajo en \emph{hitos} y estos en \emph{tareas} para facilitar dicha 
\emph{planificación}.

La categoría del objetivo planteado justifica modificaciones en la organización genérica de la memoria del trabajo. Así en el caso de estudios y validación de hipótesis el apartado de resultados y conclusiones debería incluir los resultados de experimentación y los comentarios de cómo dichos resultados validan o refutan la hipótesis planteada.


\chapter{Metodología}
\label{cap:Metodologia}

En este capítulo se debe detallar las metodologías\index{metodología} 
empleadas para planificación y desarrollo del trabajo, así como explicar de 
modo claro y conciso cómo se han aplicado dichas metodologías.

A continuación se incluye una guía rápida que puede ser de gran utilidad en la elaboración de este capítulo.

\section{Guía rápida de las metodologías de desarrollo de software}

\subsection{Proceso de desarrollo de software}

El \textbf{proceso de desarrollo de software} se denomina también 
\textbf{ciclo de vida del desarrollo del software} (\emph{SDLC, Software 
Development Life-Cycle})\index{SDLC}\index{Life-Cycle} y cubre las siguientes 
actividades:

\begin{enumerate}
\item \textbf{Obtención y análisis de requisitos}\index{requisitos} 
(\emph{requirements analysis}). Es la definición de la funcionalidad del 
software a desarrollar. Suele requerir entrevistas entre los ing. de software 
y el cliente para obtener el ¿qué y cómo? Permite obtener una 
\emph{especificación funcional} del software.

\item \textbf{Diseño} (\emph{SW design}).\index{diseño} Consiste en la 
definición de: la arquitectura, componentes, interfaces y otras 
características del sistema o sus componentes.

\item \textbf{Implementación} (\emph{SW construction and coding}). Es el
  proceso de codificación del software en un lenguaje de programación.
  Constituye la fase en que tiene lugar el desarrollo de software.

\item \textbf{Pruebas} (\emph{testing and verification}).\index{testing} 
Verificación del correcto funcionamiento del software para detectar fallos lo 
antes posible. Persigue la obtención de software de calidad. Consisten en 
pruebas de \emph{caja negra} y \emph{caja blanca}. Las primeras comprueban 
que la funcionalidad es la esperada y para ello se verifica que ante un 
conjunto amplio de entradas, la salida es correcta. Con las segundas se 
comprueba la robustez del código sometiéndolo a pruebas cuya finalidad es 
provocar fallos de software. Esta fase también incorpora la \emph{pruebas de 
integración} en las que se verifica la interoperabilidad del sistema con 
otros existentes.

\item \textbf{Documentación} (\emph{documentation}). Persigue facilitar la mejora continua del software y su mantenimiento.

\item \textbf{Despliegue} (\emph{deployment}).\index{despliegue} Consiste en 
la instalación del software en un entorno de producción y puesta en marcha 
para explotación. En ocasiones implica una fase de \emph{entrenamiento} de 
los usuarios del software.

\item \textbf{Mantenimiento} (\emph{maintenance}).\index{mantenimiento} Su 
propósito es la resolución de problemas, mejora y adaptación del software en 
explotación.
\end{enumerate}




\subsection{Metodologías de desarrollo software}
Las metodologías son el modo en que las fases del proceso software se organizan e interaccionan para conseguir que dicho proceso sea reproducible y predecible para aumentar la productividad y la calidad del software.

Una metodología es una colección de:

\begin{itemize}
\item \textbf{Procedimientos} (indican cómo hacer cada tarea y en qué momento),
\item \textbf{Herramientas} (ayudas para la realización de cada tarea), y
\item \textbf{Ayudas documentales}.
\end{itemize}

Cada metodología es apropiada para un tipo de proyecto dependiendo de sus 
características técnicas, organizativas y del equipo de trabajo. En los 
entornos empresariales es obligado, a veces, el uso de una metodología 
concreta (p.~ej. para participar en concursos públicos). El estándar 
internacional ISO/IEC 12270\index{ISO/IEC 12270} describe el método para 
seleccionar, implementar y monitorear el ciclo de vida del software.

Mientras que unos intentan sistematizar y formalizar las tareas de diseño, otros aplican técnicas de gestión de proyectos para dicha tarea. Las metodologías de desarrollo se pueden agrupar dentro de varios enfoques según se señala a continuación.

\begin{enumerate}
\item \textbf{Metodología de Análisis y Diseño de Sistemas Estructurados} 
(\emph{SSADM, Structured Systems Analysis and Design 
Methodology}).\index{SSADM} Es uno de los paradigmas más antiguos. En esta 
metodología se emplea un modelo de desarrollo en cascada 
(\emph{waterfall}).\index{modelo!waterfall@\emph{waterfall}} Las fases de 
desarrollo tienen lugar de modo secuencial. Una fase comienza cuando termina 
la anterior. Es un método clásico poco flexible y adaptable a cambios en los 
requisitos. Hace especial hincapié en la planificación derivada de una 
exhaustiva definición y análisis de los requisitos. Son metodologías que no 
lidian bien con la flexibilidad requerida en los proyectos de desarrollo 
software. Derivan de los procesos en  ingeniería tradicionales y están 
enfocadas a la reducción del riesgo. Emplea tres técnicas clave:

\begin{itemize}
\item Modelado lógico de datos (\emph{Logical Data 
Modelling}),\index{modelado}
\item Modelado de flujo de datos (\emph{Data Flow Modelling}), y
\item Modelado de Entidades y Eventos (\emph{Entity Event
  Modelling}).
\end{itemize} 

\item \textbf{Metodología de Diseño Orientado a Objetos} (\emph{OOD,  
Object-Oriented Design}).\index{OOD} Está muy ligado a la OOP (Programación 
Orientada a Objetos) en que se persigue la reutilización. A diferencia del 
anterior, en este paradigma los datos y los procesos se combinan en una única 
entidad denominada \emph{objetos} (o clases). Esta orientación pretende que 
los sistemas sean más modulares, mejorando la eficiencia y calidad del 
análisis y el diseño. Emplea extensivamente el Lenguaje Unificado de Modelado 
(UML)\index{UML} para especificar, visualizar, construir y documentar los 
artefactos de los sistemas software y  también el modelo de negocio. UML 
proporciona una serie diagramas básicos para modelar un sistema: 

\begin{itemize}
\item Diagrama de Clase (\emph{Class Diagram}). Muestra los objetos del sistema y sus relaciones. 
\item Diagrama de Caso de Uso (\emph{Use Case Diagram}). En el se plasma la
  funcionalidad del sistema y quién interacciona con él.
\item Diagrama de secuencia (\emph{Sequence Diagram}). Muestra los eventos que se
  producen en el sistema y como éste reacciona ante ellos. 
\item Modelo de Datos (\emph{Data Model}).
\end{itemize} 
                               
\item \textbf{Desarrollo Rápido de Aplicaciones} (\emph{RAD, Rapid 
Application Developmnent}).\index{RAD} Su filosofía es sacrificar calidad a 
cambio de poner en producción el sistema rápidamente con la funcionalidad 
esencial. Los procesos de especificación, diseño e implementación son 
simultáneos. No se realiza una especificación detallada y se reduce la 
documentación de diseño. El sistema se diseña en una serie de pasos, los 
usuarios evalúan cada etapa en la que proponen cambios y nuevas mejoras. Las 
interfaces de usuario se desarrollan habitualmente mediante sistemas 
interactivos de desarrollo. En vez de seguir un modelo de desarrollo en 
cascada sigue un modelo en espiral (Boehm).\index{modelo!espiral} La clave de 
este modelo es el desarrollo continuo que ayuda a minimizar los riesgos. Los 
desarrolladores deben definir las características de mayor prioridad. Este 
tipo de desarrollo se basa en la creación de prototipos y realimentación 
obtenida de los clientes para definir e implementar más características hasta 
alcanzar un sistema aceptable para despliegue.

\item \textbf{Metodologías Ágiles}. \emph{"[...] envuelven un enfoque para la 
toma de decisiones en los proyectos de software, que se refiere a métodos de 
ingeniería del software basados en el desarrollo iterativo e incremental, 
donde los requisitos y soluciones evolucionan con el tiempo según la 
necesidad del proyecto. Así el trabajo es realizado mediante la colaboración 
de equipos auto-organizados y multidisciplinarios, inmersos en un proceso 
compartido de toma de decisiones a corto plazo. Cada iteración del ciclo de 
vida incluye:  planificación, análisis de requisitos, diseño, codificación, 
pruebas y  documentación. Teniendo gran importancia el concepto de 
"Finalizado" (Done), ya que el objetivo de cada iteración no es agregar toda 
la funcionalidad para justificar el lanzamiento del producto al mercado, sino 
incrementar el valor por medio de "software que funciona" (sin errores). Los 
métodos ágiles enfatizan las comunicaciones cara a cara en vez de la 
documentación. [...]"}\footnote{Wikipedia}\index{Wikipedia}
\end{enumerate}

\subsection{Proceso de testing}\index{testing}

\begin{enumerate}
\item \emph{Pruebas modulares}. (pruebas unitarias) De este modo se intenta hacer pruebas sobre un módulo tan pronto como sea posible. Las \emph{pruebas unitarias} que comprueban el correcto funcionamiento de una unidad de código. Dicha unidad elemental de código consistiría en cada función o procedimiento, en el caso de programación estructurada y cada clase, para la programación orientada a objetos. Las características de una prueba unitaria de calidad son: \emph{automatizable} (sin intervención manual), \emph{completa},  \emph{reutilizable}, \emph{independiente} y \emph{profesional}.

\item \emph{Pruebas de integración}. Pruebas de varios módulos en conjunto para comprobar su interoperabilidad.

\item \emph{Pruebas de caja negra}.

\item \emph{Beta testing}.

\item \emph{Pruebas de sistema y aceptación}.

\item \emph{Training}.
\end{enumerate}






\subsection{Herramientas CASE (\emph{Computer Aided Software Engineering})}

Las herramientas CASE\index{CASE} están destinadas a facilitar una o varias 
de las tareas implicadas en el ciclo de vida del desarrollo de software. Se 
pueden dividir en la siguientes categorías:

\begin{enumerate}
\item Modelado y análisis de negocio.
\item Desarrollo. Facilitan las fases de diseño y construcción.
\item Verificación y validación.
\item Gestión de configuraciones.
\item Métricas y medidas.
\item Gestión de proyecto. Gestión de planes, asignación de tareas, planificación, etc.
\end{enumerate}




\subsubsection{IDE (Integrated Development Environment)}\index{IDE}
\begin{multicols}{2}
\begin{itemize}
\item \href{https://notepad-plus-plus.org/}{Notepad++}
\item \href{https://code.visualstudio.com/}{Visual Studio Code}
\item \href{https://atom.io/}{Atom}
\item \href{https://www.gnu.org/s/emacs/}{GNU Emacs}
\item \href{https://netbeans.org/}{NetBeans}
\item \href{https://eclipse.org/}{Eclipse}
\item \href{https://www.qt.io/ide/}{Qt Creator}
\item \href{http://www.jedit.org/}{jEdit}
\item \href{https://www.jetbrains.com/idea/}{ItelliJ IDEA}
\end{itemize}
\end{multicols}



\subsubsection{Depuración}\index{depuración}
\begin{itemize}
\item \href{https://www.gnu.org/s/gdb/}{GNU Debugger}
\end{itemize}


\subsubsection{Testing}\index{testing}
\begin{multicols}{2}
\begin{itemize}
\item \href{http://junit.org}{JUnit}. Entorno de pruebas para Java.
\item \href{http://cunit.sourceforge.net/}{CUnit}. Entorno de pruebas para C.
\item \href{https://wiki.python.org/moin/PyUnit}{PyUnit}. Entorno de pruebas para Python.
\end{itemize}
\end{multicols}

\subsubsection{Repositorios y control de versiones}\index{control de 
versiones}
\begin{multicols}{2}
\begin{itemize}
\item \href{https://git-scm.com/}{Git}
\item \href{https://www.mercurial-scm.org/}{Mercurial}
\item \href{https://github.com/}{Github}
\item \href{https://bitbucket.org/}{Bitbucket}
\item \href{https://www.sourcetreeapp.com/}{SourceTree}
\end{itemize}
\end{multicols}


\subsubsection{Documentación}
\begin{multicols}{2}
\begin{itemize}
\item \href{https://www.latex-project.org/}{\LaTeX}
\item \href{https://markdown.es/}{Markdown}
\item \href{http://www.stack.nl/\%7Edimitri/doxygen/index.html}{Doxygen}
\item \href{http://mtmacdonald.github.io/docgen/docs/index.html}{DocGen}
\item \href{http://pandoc.org/}{Pandoc}
\end{itemize}
\end{multicols}



\subsubsection{Gestión y planificación de proyectos}\index{planificación}
\begin{multicols}{2}
\begin{itemize}
\item \href{https://trello.com/}{Trello}
\item \href{https://es.atlassian.com/software/jira}{Jira}
\item \href{https://asana.com/}{Asana}
\item \href{https://slack.com/}{Slack}
\item \href{https://basecamp.com/}{Basecamp}
\item \href{https://www.teamwork.com/project-management-software}{Teamwork Projects}
\item \href{https://www.zoho.com/projects/}{Zoho Projects}
\end{itemize}
\end{multicols}


\subsection{Fuentes de información adicional}
\begin{itemize}
\item \href{https://leankit.com/blog/2019/03/top-6-software-development-methodologies/}{Top
  6 Software Development Methodologies}. Maja Majewski. Planview
  LeanKit, 2019.
\item \href{https://acodez.in/12-best-software-development-methodologies-pros-cons/}{12
  Best software development methodologies with pros and cons}. acodez,
  2018.
\item \href{http://www.itinfo.am/eng/software-development-methodologies/}{Software
  Development Methodologies}. Association of Modern Technologies
  Professionals, 2019.
\end{itemize}

\chapter{Resultados}
\label{cap:Resultados}

En esta sección se describirá la aplicación del método de trabajo presentado en el capítulo \ref{cap:Metodologia} en este caso concreto, mostrando los elementos (modelos, diagramas, especificaciones, etc.) más importantes. Este apartado debe explicar cómo la metodología satisface los objetivos y requisitos planteados.
\chapter{Conclusiones}
\label{cap:Conclusiones}

En este capítulo se realizará un juicio crítico y discusión sobre los resultados obtenidos. \emph{Cuidado, esta discusión no debe confundirse con una valoración del enriquecimiento personal que supone la realización del trabajo como culminación de una etapa académica}. Aunque de gran importancia, esta última valoración debe quedar fuera de la memoria del trabajo y solo debe ahondarse en ella ante requerimiento explícito del comité en el acto de defensa.

Si es pertinente deberá incluir información sobre trabajos derivados como publicaciones o ponencias en preparación, así como trabajos futuros \emph{(solo si estos están iniciados o planificados en el momento que se redacta el texto)}. Evitar hacer una lista de posibles mejoras. Contrariamente a lo que alguno pueda pensar generalmente aportan impresión de trabajo incompleto o inacabado.\footnote{Puede reflexionarse en ello por si en la defensa del trabajo se pregunta sobre estas posibles mejoras.}


\section{Justificación de competencias adquiridas}
Es muy importante recordar que según la normativa vigente en la ESI, el capítulo de conclusiones debe incluir \emph{obligatoriamente} un apartado destinado a justificar la aplicación en el TFG de competencias específicas (una o más) adquiridas en la tecnología específica cursada.

En el TFG se han aplicado las competencias correspondientes a la Tecnología Específica de \emph{[poner lo que corresponda]}:

\begin{description}
\item[Código de la competencia 1:] \emph{[Texto de la competencia 1]}. Explicación de cómo se ha aplicado en el TFG.
\item[\dots] otras más si las hubiera.
\end{description}







% -------------------------


%--- BACKMATTER
%\backmatter (se comenta para que los índices puedan aparecer después de la bibliografía)



% -------------------------
%
% BIBLIOGRAFÍA
%
% -------------------------
% OJO: Todas las referencias deben estar citadas en el texto)
% EDITAR: Comentar línea siguiente
\nocite{*} % INCLUIDO para ver cómo queda, pero comentar en versión final.

\cleardoublepage % OJO: Necesario para ajustar el avance de página
\phantomsection  % OJO: Ojo necesario con hyperref.
\addcontentsline{toc}{chapter}{\bibname} % Añade la bibliografía al Índice de contenidos.
%---
% Opción 1: Bibliografía con todas las fuentes en un apartado.
%---
\printbibliography
%---

%---
% Opción 2: Bibliografía con secciones separadas.
%---
%\printbibheading
%\printbibliography[heading=subbibliography,type=online,title={Fuentes online}]
%\printbibliography[heading=subbibliography,nottype=online,title={Fuentes no online}]
% -------------------------


% -------------------------
%
% OPT.: ANEXOS: Comentar si no se desean incluir.
% Mover si se desea que aparezcan antes de la bobliografía.
%
% -------------------------
\appendix
\portadaAnexos % OPT. Añade una portada para anexos

% Tras este punto los capítulos se numeran con letras.
% Aquí todos los apéndices necesarios
\chapter{Sobre la Bibliografía}
\label{cap:AnexoA}

En los anexos se incluirá, de modo opcional, material suplementario que podrá consistir en manuales de usuario, listados seleccionados de código fuente, esquemas, planos y en general aquel contenido que complementa a la memoria. Se recomienda que no sean excesivamente voluminosos, aunque su extensión no está sometida a la regulación por normativa, ya que esta afecta únicamente al texto principal de la memoria.

En esta plantilla hemos decidido incluir dos anexos. En el primero de ellos se hacen algunos comentarios adicionales sobre la bibliografía. En el segundo se aporta una breve introducción a \LaTeX{} cuya información puede servir de ejemplo de inclusión de ciertos elementos en la preparación de la memoria del TFG.

Todo el material de terceros se debe citar convenientemente sin contravenir los términos de las licencias de uso y distribución de dicho material. Esto se extiende al uso de diagramas y cualquier elemento gráfico. El incumplimiento de la legislación vigente en materia de protección de la propiedad intelectual es responsabilidad exclusiva del autor, independientemente de la cesión de derechos que este haya convenido.\footnote{\url{https://www.uclm.es/areas/biblioteca/encuentra-informacion/perfiles/alumno/antiplagio}}

La sección de \emph{Bibliografía}, que si se prefiere se puede titular \emph{Referencias}, incluirá un listado ordenado preferentemente por orden alfabético (primer apellido del autor principal), con todas las obras citadas en el texto. En la lista de referencias se especificará para cada obra: 
\begin{itemize}
	\item autores, 
	\item título, 
	\item editorial, y 
	\item año de publicación.
\end{itemize} 

Este formato se conseguirá en \LaTeX{} mediante el uso del estilo estándar \texttt{plain} o cualquier otro derivado con estilo de citación numérica. En algunas titulaciones se obliga a una ordenación por orden de cita en el texto. Este resultado se obtiene con Bib\TeX{} mediante los estilos estándar \texttt{ieeetr} (estilo para los IEEE \emph{transactions}) y \texttt{unsrt} (estilo \emph{unsorted}).\footnote{\url{https://www.wikiwand.com/es/articles/BibTeX}}

Es muy importante tener presente que en esta sección solo se debe incluir las referencias bibliográficas citadas expresamente en el documento. Si se desea incluir fuentes consultadas, pero no citadas, se puede confeccionar con ellas una sección denominada \emph{Material de consulta}, aunque estas referencias se pueden incluir opcionalmente a lo largo del documento como notas a pie de página.

En las titulaciones técnicas se empleará estilo de citación numérico con el número de la referencia entre corchetes. La cita podrá incluir el número de página concreto de la referencia que se desea citar. El uso correcto de la citación implica dejar claro al lector cuál es el texto, material o idea citado. .

Cuando se desee incluir referencias a páginas genéricas de Internet sin mención expresa a un artículo con título y autor definido, dichas referencias se pueden incluir como notas al pie de página o como un apartado de fuentes de consulta dedicado a \emph{Direcciones de Internet}. Por el contrario, los documentos electrónicos publicados en Internet se pueden incluir en la sección de Bibliografía empleando el tipo de entrada \texttt{misc} en el fichero \texttt{.bib} con el comando \texttt{url} (como se muestra en la bibliografía de ejemplo distribuida con este documento). Observarás que el campo \texttt{note} se emplea para añadir información adicional como la fecha de la última consulta de fuentes publicadas en Internet, y para la inclusión del DOI de algunas obras para su rápida recuperación.\footnote{Esta estrategia necesita adaptación cuando se emplea un estilo de citación autor-año (p.ej., \texttt{natbib}).}







 % Apéndice A (opcionales)

%---



% -------------------------
%
% OPT.: ÍNDICE TEMÁTICO: Comentar si no se desean incluir.
%
% -------------------------
% CONSEJO: Incluir los comandos mientras se escribe cada capítulo ya que hacerlo al final resulta tedioso.
\cleardoublepage % OJO: Necesario para ajustar el avance de página
\phantomsection  % OJO: Ojo necesario con hyperref.
% EDITAR: Si se desea cambiar el nomre del ínidice temático
%\renewcommand{\indexname}{Índice de acrónimos}
\addcontentsline{toc}{chapter}{\indexname} % Añade al Índice de contenidos.
\printindex  % Facilitado por makeidx (opcional, si no se usa no se imprime)
%---
\end{document}