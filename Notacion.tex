% -------------------------
%
% -NOTACIÓN: Lista de símbolos con significado especial.
%
% -------------------------
\cleardoublepage
\phantomsection % OJO: Necesario con hyperref

\chapter*{Notación} % Opción con * para que no aparezca en TOC ni numerada
\addcontentsline{toc}{chapter}{Notación} % Añade al TOC.

Ejemplo de lista con notación (o nomenclatura) empleada en la memoria del TFG.\footnote{Se incluye unicamente con propósito de ilustración, ya que el documento no emplea la notación aquí mostrada.}

\begin{tabular}{r r p{0.8\linewidth}}
$A, B, C, D$	& : & Variables lógicas. \\
$f, g, h$		& :	& Funciones lógicas. \\
$\cdot$			& : & Producto lógico (AND). A menudo se omitirá como en $A 
B$ en lugar de $A \cdot B$.\\
$+$				& : & Suma aritmética o lógica (OR) dependiendo del 
contexto.\\
$\oplus$		& : & OR exclusivo (XOR).\\
$\overline{A}$ o ${A}'$	& : & Operador NOT o negación.
\end{tabular}