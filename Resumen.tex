%--- Ajustes del documento.
\pagestyle{plain}	% Páginas sólo con numeración inferior al pie

% -------------------------
%
% RESUMEN:
% OJO: Si es preciso cambiar manualmente orden Resumen <-> Abstract
%
% -------------------------
%--- Resumen en español
\selectlanguage{spanish} % Selección de idioma del resumen.
\cleardoublepage % Se incluye para modificar el contador de página antes de añadir bookmark
\phantomsection  % OJO: Necesario con hyperref
\pdfbookmark[0]{Resumen}{idx_resumen}% idx_resumen.0 % Bookmark en PDF
\addcontentsline{toc}{chapter}{Resumen} % Añade al TOC.

\begin{abstract}
% EDITAR: Resumen (máx. 1 pág.)
\begin{center}
\emph{(... versión del resumen en español ...)}
\end{center}
El resumen debe ocupar como máximo una página y en dicho espacio proporcionará información crucial sobre el \emph{`qué'} (problemática que trata de resolver el TFG), el \emph{`cómo'} (metodología para llegar a los resultados) y los objetivos alcanzados.
\end{abstract}
%---



%--- Resumen en inglés
% Abstract
\selectlanguage{english} % Selección de idioma del resumen.
\cleardoublepage
\phantomsection % OJO: Necesario con hyperref
\pdfbookmark[0]{Abstract}{idx_abstract}% idx_abstract.0 % Bookmark en PDF

\begin{abstract}
% EDITAR: Abstract (máx. 1 pág.)

\begin{center}
\emph{(... english version of the abstract ...)}
\end{center}
Versión del resumen en inglés. En los trabajos cuyo idioma principal sea el inglés, el orden de \textsf{Resumen} y \textsf{Abstract} se invertirá.	
\end{abstract}
%---

%--- Ajuste del idioma para el resto del documento.
\ifspanish
	\selectlanguage{spanish}% Emplea idioma español
\else
	\selectlanguage{english}% Emplea idioma inglés
\fi
