\chapter{Plan de gestión del trabajo}\label{cap:Planificacion}
En este capítulo se debe detallar todos los aspectos relacionados con el plan de gestión del trabajo que incluyen: 
\begin{itemize}[noitemsep]
\item la metodología de desarrollo, 
\item las tecnologías y recursos necesarios, 
\item la gestión de la configuración y el aseguramiento de la calidad, 
\item la planificación del trabajo, y
\item la estimación de costes y el análisis de riesgos.
\end{itemize}




\section{Guía rápida de las metodologías de desarrollo de software}
El \textbf{proceso de desarrollo de software} se denomina también \textbf{ciclo de vida del desarrollo del software} (\emph{SDLC, Software Development Life-Cycle}) y cubre las siguientes actividades:

\begin{enumerate}[1.-]
\item \textbf{Obtención y análisis de requisitos} (\emph{requirements analysis}). Es la definición de la funcionalidad del software a desarrollar. Suele requerir entrevistas entre los ingenieros de software y el cliente para obtener el `\textsc{qué}' y `\textsc{cómo}'. Permite obtener una \emph{especificación funcional} del software.

\item \textbf{Diseño} (\emph{SW design}). Consiste en la definición de la arquitectura, los componentes, las interfaces y otras características del sistema o sus componentes.

\item \textbf{Implementación} (\emph{SW construction and coding}). Es el proceso de codificación del software en un lenguaje de programación.  Constituye la fase en que tiene lugar el desarrollo de software.

\item \textbf{Pruebas} (\emph{testing and verification}).Verificación del correcto funcionamiento del software para detectar fallos lo antes posible. Persigue la obtención de software de calidad. Consisten en pruebas de \emph{caja negra} y \emph{caja blanca}. Las primeras comprueban que la funcionalidad es la esperada y para ello se verifica que ante un conjunto amplio de entradas, la salida sea correcta. Con las segundas se comprueba la robustez del código sometiéndolo a pruebas cuya finalidad es provocar fallos de software. Esta fase también incorpora la \emph{pruebas de integración} en las que se verifica la interoperabilidad del sistema con otros existentes.

\item \textbf{Documentación} (\emph{documentation}). Persigue facilitar la mejora continua del software y su mantenimiento.

\item \textbf{Despliegue} (\emph{deployment}). Consiste en la instalación del software en un entorno de producción y puesta en marcha para explotación. En ocasiones implica una fase de \emph{entrenamiento} de los usuarios del software.

\item \textbf{Mantenimiento} (\emph{maintenance}). Su propósito es la resolución de problemas, mejora y adaptación del software en explotación.
\end{enumerate}





\subsection{Metodologías de desarrollo software}
\emph{Las metodologías son el modo en que las fases del proceso de desarrollo de software se organizan e interaccionan para conseguir que dicho proceso sea reproducible y predecible para aumentar la productividad y la calidad del software.}

\noindent Una metodología es una colección de:

\begin{enumerate}[A.]
\item \textbf{Procedimientos} (indican cómo hacer cada tarea y en qué momento),
\item \textbf{Herramientas} (ayudas para la realización de cada tarea), y
\item \textbf{Ayudas documentales}.
\end{enumerate}

Cada metodología es apropiada para un tipo de proyecto dependiendo de sus características técnicas, organizativas y del equipo de trabajo. En los entornos empresariales es obligado, a veces, el uso de una metodología concreta (p.~ej., para participar en concursos públicos). El estándar internacional ISO/IEC 12270 describe el método para seleccionar, implementar y monitorear el ciclo de vida del software~\cite{Vijayasarathy16,Appelbaum22}.

Mientras que unas metodologías intentan sistematizar y formalizar las tareas de diseño, otras aplican técnicas de gestión de proyectos para dicha tarea. Las metodologías de desarrollo se pueden agrupar dentro de varios enfoques según se resume a continuación en la tabla~\ref{tab:metodologias}.

\begin{table}[H]
	\centering
	\caption{Resumen de metodologías de desarrollo de software}
	\label{tab:metodologias}
	\begin{tabular}{|>{\raggedright\arraybackslash}p{3.2cm}|p{4.5cm}|p{6.5cm}|}
	\hline
	\textbf{Metodología} & \textbf{Características} & \textbf{Herramientas/Enfoques asociados} \\
	\hline
	\textbf{SSADM} (\emph{Structured Systems Analysis and Design Methodology}) & Metodología estructurada, secuencial (modelo en cascada), centrada en análisis exhaustivo de requisitos. & Modelado lógico de datos, flujo de datos, entidades y eventos. \\
	\hline
	\textbf{OOD} (\emph{Object-Oriented Design}) & Centrado en la modularidad y reutilización del código a través de objetos y clases. & \textbf{UML} (Diagramas de clase, casos de uso, secuencia, modelo de datos). \\
	\hline
	\textbf{RAD} (\emph{Rapid Application Development}) & Desarrollo iterativo y rápido basado en prototipos, minimiza documentación, enfatiza entrega rápida. & Interfaces gráficas, prototipado, ciclo en espiral (Boehm), evaluación continua por parte del usuario. \\
	\hline
	Metodologías Ágiles & Desarrollo incremental e iterativo, énfasis en la colaboración y en software funcional, poca documentación. & \emph{Scrum}, \emph{Kanban}, reuniones diarias, sprints, enfoque en el concepto de “hecho (\emph{done})”. \\
	\hline
	\end{tabular}
\end{table}






\subsection{Proceso de \emph{testing}}

El testeo del software puede consistir en varios tipos de procesos:
\begin{enumerate}[noitemsep]
\item \emph{Pruebas modulares} (pruebas unitarias). Su propósito es hacer pruebas sobre un módulo tan pronto como sea posible. Las \emph{pruebas unitarias} comprueban el correcto funcionamiento de una unidad de código. En la programación estructurada, dicha unidad elemental de código consistiría en cada función o procedimiento. Para la programación orientada a objetos se trataría de cada clase. Las características de una prueba unitaria de calidad son: \emph{automatizable} (sin intervención manual), \emph{completa},  \emph{reutilizable}, \emph{independiente} y \emph{profesional}.

\item \emph{Pruebas de integración}. Pruebas de varios módulos en conjunto para comprobar su interoperabilidad.

\item \emph{Pruebas de caja negra}.

\item \emph{Beta testing}.

\item \emph{Pruebas de sistema y aceptación}.

\item \emph{Training}.
\end{enumerate}



\section{Otros aspectos del plan de gestión}
Además de la metodología se deben abordar los aspectos siguientes relacionados con el plan de gestión del proyecto:
\begin{itemize}
\item \textbf{Recursos}. En este apartado se describirán los recursos hardware y software empleados. También debería quedar aclarado el número y papel de los integrantes del equipo de proyecto.

\item \textbf{Gestión de la configuración y aseguramiento de la calidad}. Durante el desarrollo de proyectos de software es esencial definir la estrategia de control de versiones y las diferentes \emph{releases}. Además, en esta sección se describirán las actividades y tareas que garantizan la calidad del proceso de desarrollo del software, incorporando los estándares, prácticas y normas de aplicación. También se deben documentar los distintos tipos de revisiones, verificaciones y validaciones que se realizarán, criterios de aprobación o rechazo de cada, y los procedimientos para llevar a cabo acciones correctivas o preventivas.

\item \textbf{Planificación}. Detallará la estimación de la evolución temporal del proyecto, marcando sus iteraciones e hitos básicos. Para ello, se emplearán diagramas Gantt y debería incluir una comparación cuantitativa del tiempo y el esfuerzo realmente invertido frente al estimado.

\item \textbf{Costes}. Análisis y presupuesto del coste de los recursos (humanos y materiales) necesarios para el proyecto. El cálculo de costes de personal debe tener en cuenta la realidad del mercado laboral en España. Dicho cálculo se puede hacer en persona/mes, y luego hacer la correspondencia al coste monetario. 

\item \textbf{Análisis de riesgos}. En esta sección se debería incluir una enumeración de los riesgos del proyecto, indicando su posible impacto (efecto que la ocurrencia del citado riesgo tendría en el desarrollo del proyecto) y la probabilidad de ocurrencia. Una vez se identifican los riesgos, se deben priorizar para definir los planes necesarios que reduzcan su impacto o incluso su probabilidad de ocurrencia.
\end{itemize}


\section[Tecnologías]{Revisión de tecnologías y herramientas CASE (\emph{Computer Aided Software Engineering})}
Además de los recursos humanos y de hardware necesarios en el trabajo, en la sección de recursos software se deberían enumerar las herramientas software previstas. A continuación se realiza una revisión rápida no exhaustiva de las herramientas más populares.\footnote{Este listado, no exhaustivo, es meramente informativo, ya que en la memoria de un TFG solo se deben incluir aquellas que se hayan evaluado y empleado finalmente.}

Las herramientas CASE están destinadas a facilitar una o varias de las tareas implicadas en el ciclo de vida del desarrollo de software. Se pueden dividir en las siguientes categorías:

\begin{enumerate}[noitemsep]
\item Modelado y análisis de negocio.
\item Desarrollo. 
\item Verificación y validación.
\item Gestión de configuraciones.
\item Métricas y medidas.
\item Gestión de proyecto (gestión de planes, asignación de tareas, planificación, etc.).
\end{enumerate}

\subsection{IDE (Integrated Development Environment)}
\begin{multicols}{2}
\begin{itemize}[nosep]
%\item \href{https://notepad-plus-plus.org/}{Notepad++}
\item \href{https://code.visualstudio.com/}{Visual Studio Code}
\item \href{https://atom.io/}{Atom}
\item \href{https://www.gnu.org/s/emacs/}{GNU Emacs}
\item \href{https://netbeans.org/}{NetBeans}
\item \href{https://eclipse.org/}{Eclipse}
\item \href{https://www.qt.io/ide/}{Qt Creator}
\item \href{http://www.jedit.org/}{jEdit}
\item \href{https://www.jetbrains.com/idea/}{ItelliJ IDEA}
\end{itemize}
\end{multicols}

\subsection{Depuración}
\begin{itemize}[nosep]
\item \href{https://www.gnu.org/s/gdb/}{GNU Debugger}
\end{itemize}

\subsection{Testing}
\begin{multicols}{2}
\begin{itemize}[nosep]
\item \href{http://junit.org}{JUnit}. Entorno de pruebas para Java.
\item \href{http://cunit.sourceforge.net/}{CUnit}. Entorno de pruebas para C.
\item \href{https://wiki.python.org/moin/PyUnit}{PyUnit}. Entorno de pruebas para Python.
\item \href{https://nunit.org/}{NUnit}. Entorno de pruebas para .Net.
\end{itemize}
\end{multicols}

\subsection{Repositorios y control de versiones}
\begin{multicols}{2}
\begin{itemize}[nosep]
\item \href{https://git-scm.com/}{Git}
\item \href{https://github.com/}{Github}
\item \href{https://www.mercurial-scm.org/}{Mercurial}
\item \href{https://bitbucket.org/}{Bitbucket}
\item \href{https://www.sourcetreeapp.com/}{SourceTree}
\end{itemize}
\end{multicols}


\subsection{Documentación}
\begin{multicols}{2}
\begin{itemize}[nosep]
\item \href{https://www.latex-project.org/}{\LaTeX}
\item \href{https://www.overleaf.com/}{Overleaf}
\item \href{https://markdown.es/}{Markdown}
\item \href{http://www.stack.nl/\%7Edimitri/doxygen/index.html}{Doxygen}
\item \href{http://mtmacdonald.github.io/docgen/docs/index.html}{DocGen}
\item \href{http://pandoc.org/}{Pandoc}
\end{itemize}
\end{multicols}


\subsection{Gestión y planificación de proyectos}
\begin{multicols}{2}
\begin{itemize}[nosep]
\item \href{https://trello.com/}{Trello}
\item \href{https://es.atlassian.com/software/jira}{Jira}
\item \href{https://asana.com/}{Asana}
\item \href{https://slack.com/}{Slack}
\item \href{https://basecamp.com/}{Basecamp}
\item \href{https://www.teamwork.com/project-management-software}{Teamwork Projects}
\item \href{https://www.zoho.com/projects/}{Zoho Projects}
\end{itemize}
\end{multicols}


