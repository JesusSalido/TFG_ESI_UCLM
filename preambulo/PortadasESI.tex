% OJO: Editar este fichero a voluntad para ajustarlo al resultado deseado.
% OJO: Poner especial atención en usar los adjetivos de género correctos.
% -------------------------
% -------------------------
% -------------------------
% PORTADAS: (Incluidas páginas para créditos y dedicatoria)
% -------------------------
% -------------------------------------------------------------------------
% -------------------------------------------------------------------------
% -------------------------------------------------------------------------
% PORTADA PRAL. (1)
% Puedes incluir directamente la portada realizada por un programa externo.
% Y directamente modificar el diseño que se proporciona. 
% NOTA: Para eliminar líneas sin alterar el espaciado original se recomienda emplear \phantom{text}
\begin{titlepage}
    \makeatletter
	\begin{center}
        \pdfbookmark{Portada}{portada}
        \vspace{1cm}
		\includegraphics[width=4.5cm]{\@escudo}\vspace{1cm} 
		
		{\LARGE \textbf{\@instEdu\\[0.5ex]
				\@centroEdu\\[2cm]
				\@titulacion}}\\[0.5cm]
        {\large \textbf{Especialidad cursada}}\\[1.5cm] 
        % (EN: Specialization in ...)
		{\LARGE \textbf{\@tipoDoc}}\\[1cm]	
		{\LARGE \@tituloPrimera}\\ \smallskip%			
		\ifdefined\@tituloSegunda{\LARGE \@tituloSegunda}\\[3cm]
		\else \phantom{\LARGE	Texto fantasma}\\[3cm]
		\fi
		{\Large \@autor}\vfill%
	\end{center}
	
	\begin{flushright}
		{\Large \ifspanish \@mesTF \else \@monthTF \fi, \@yearTF}
	\end{flushright}
	
	\cleardoublepage
    \makeatother
\end{titlepage}






% -------------------------------------------------------------------------
% -------------------------------------------------------------------------
% -------------------------------------------------------------------------
% PORTADA INTERIOR (2)
% Puedes incluir directamente la portada realizada por un programa externo.
% Y directamente modificar el diseño que se proporciona. 
% NOTA: Para eliminar líneas sin alterar el espaciado original se recomienda emplear \phantom{text}
\begin{titlepage}
    \makeatletter
	\begin{center}
        \vspace{1cm}
		\includegraphics[width=4.55cm]{\@escudo}\vspace{1cm}
		
		{\LARGE \textbf{\@instEdu \\[0.5ex]
				\@centroEdu}}\\[0.5cm]
		{\Large \textbf{Depto. del Tutor}}\\ \smallskip%
        {\Large\textbf{(cont.)}}\\[0.5cm]
		{\large \textbf{Especialidad cursada}}\\[1.5cm]
%		(EN: Specialization in ...)
		{\LARGE \textbf{\@tipoDoc}}\\[1cm]
		
		
		{\LARGE \textbf{\@tituloPrimera}}\\ \smallskip%		
		\ifdefined\@tituloSegunda{\LARGE \textbf{\@tituloSegunda}}
		\else \phantom{\LARGE Texto fantasma}
		\fi
	\end{center}
	\vfill%
	\begin{flushleft}
		{\Large Autor: \@autor} \\ \bigskip% (EN: Author)
		{\Large Tutor(a): nombre y apellidos} \\ \bigskip% (EN: Co-Supervisor)
		{\Large Co-tutor(a): nombre y apellidos}
	\end{flushleft}
	\vspace{2cm}%
	\begin{flushright}
		{\Large \ifspanish \@mesTF \else \@monthTF \fi, \@yearTF}
	\end{flushright}
	\cleardoublepage
    \makeatother
\end{titlepage}
	




% -------------------------------------------------------------------------
% -------------------------------------------------------------------------
% -------------------------------------------------------------------------
% OPT.: CRÉDITOS (aunque no es obligatorio es recomendable).
% -------------------------
%
% CRÉDITOS
%
% -------------------------
% EDITAR (opcional): Licencia (si se desea modificar).
% Este comando permite una gran flexibilidad y la ventaja de no depender de paquetes externos.
% Esta es una página reservada para señalar información relativa a los derechos de autor y la licencia de distribución y uso del documento. Esta página debería ser aprovechada también para informar de cualquier tipo de cesión de los derechos anteriormente citados. El autor del TFG debe tener presente que el incumplimiento de la legislación vigente en materia de protección de la propiedad intelectual es de su exclusiva responsabilidad independientemente de la cesión de derechos que este haya convenido para su obra ya que no son objeto de cesión aquellos derechos de los que no se es poseedor.

\creditos{Este documento se distribuye con licencia Creative Commons Atribución Compartir Igual 4.0. El texto completo de la licencia puede obtenerse en \url{https://creativecommons.org/licenses/by-sa/4.0/}. 
% El escudo de Informática basado en el núcleo de ferrita que acompaña la distribución de esta plantilla ha sido realizado por P.~Moya, D.~Villa e I. Díez, su inclusión debe respetar los derechos de autor y las licencias a las que se vea sometido. 

La copia y distribución de esta obra está permitida en todo el mundo, sin regalías y por cualquier medio, siempre que esta nota sea preservada. Se concede permiso para copiar y distribuir traducciones de este libro desde el español original a otro idioma, siempre que la traducción sea aprobada por el autor del libro y tanto el aviso de copyright como esta nota de permiso, sean preservados en todas las copias.}{cclicense}
%---



%---




% -------------------------------------------------------------------------
% -------------------------------------------------------------------------
% -------------------------------------------------------------------------
% CALIFICACIÓN TRIBUNAL
% Puedes incluir directamente la portada realizada por un programa externo.
%---
\begin{titlepage}
    \pdfbookmark[1]{Tribunal}{tribunal}
    \makeatletter
	{\flushright \LARGE \textsc{Tribunal:}}
	
	\vspace*{\stretch{0.5}}
	\hspace*{1cm}{\Large Presidente: \hrulefill}
	
	\vspace*{\stretch{0.5}}
	\hspace*{1cm}{\Large Vocal: \hrulefill}
	
	\vspace*{\stretch{0.5}}
	\hspace*{1cm}{\Large Secretario(a): \hrulefill}
	
	\vspace*{\stretch{0.5}}
	{\flushright \LARGE \textsc{Fecha de defensa:} \hrulefill}
	
	\vspace*{\stretch{1.5}}
	{\flushright \LARGE \textsc{Calificación:} \hrulefill}
	
	\vspace*{\stretch{2.5}}
	\begin{center}
		\begin{tabularx}{\linewidth}{X X X}
			{\large \textsc{Presidente}} & {\large \textsc{Vocal}} & {\large \textsc{Secretario(a)}}\\[2.5cm]
			Fdo.: & Fdo.: & Fdo.:		
		\end{tabularx}
	\end{center}
	\cleardoublepage
    \makeatother
\end{titlepage}



% -------------------------------------------------------------------------
% -------------------------------------------------------------------------
% -------------------------------------------------------------------------
% OPT.: DEDICATORIA (1 pág. máximo) comentar si no se desea incluir.
% Aunque opcional, no se debería perder la oportunidad de poder 
% dedicar el trabajo a alguien MUY especial. Debe ocupar como mucho dos líneas
% (no confundir con los agradecimientos).
% EDITAR: Dedicatoria.

\null\vspace{\stretch{1}}
\begin{flushright}
\emph{A mis estudiantes \\ % A alguien muy especial
Por contribuir a hacer de cada día un reto ilusionante}
\end{flushright}
\vspace{\stretch{2}}\null
\cleardoublepage


%---
%
% FIN PORTADAS: 
% |
% |
% -> ----------------------