%--- Ajustes del documento.
\pagestyle{plain}	% Páginas sólo con numeración inferior al pie

% -------------------------
%
% RESUMEN:
% 
%

% EDITAR: Resumen (máx. 1 pág.)
%\cleardoublepage % Se incluye para modificar el contador de página antes de añadir bookmark
\phantomsection  % Necesario con hyperref
\addcontentsline{toc}{chapter}{Resumen} % Añade al TOC.
\selectlanguage{spanish} % Selección de idioma del resumen.
\makeatletter
\begin{center} %
   {\textsc{TRABAJO FIN DE GRADO - ESCUELA SUP. DE INFORMÁTICA (UCLM)}\par} % Tipo de trabajo
   \vspace{1cm} %  
   {\textbf{\Large\@tituloCorto}\par}  % Título del trabajo
   \vspace{0.4cm} %
   {\@autor \\ \@cityTF,{} \@mesTF{} \@yearTF\par} 
   \vspace{0.9cm} %
   {\textbf{\large\textsf{Resumen}}\par} % Título de resumen
\end{center}   
\makeatother %
En una página como máximo, el resumen explicará de modo breve la problemática que trata de resolver el trabajo \emph{(el `qué')}, la metodología para  abordar su solución  \emph{(el `cómo')} y los resultados obtenidos. En los trabajos cuyo idioma principal sea el inglés, el orden de \textsf{Resumen} y \textsf{Abstract} se invertirá.

En concreto este documento debe servir como guía para preparar, con LaTeX, el TFG en la \href{http://webpub.esi.uclm.es/}{Escuela Superior de Informática} (ESI) de la Univ. de Castilla-La Mancha (UCLM) siguiendo la \href{https://pruebasaluuclm.sharepoint.com/sites/esicr/tfg/SitePages/Inicio.aspx}{normativa de aplicación}. Está disponible en \href{https://github.com/JesusSalido/TFG_ESI_UCLM}{GitHub} y \href{https://www.overleaf.com/latex/templates/plantilla-de-tfg-escuela-superior-de-informatica-uclm/phjgscmfqtsw}{Overleaf}. Por tanto, puede emplearse tanto en modo local en un equipo con LaTeX instalado (\href{https://miktex.org/}{MiKTeX}, \href{https://www.tug.org/texlive/}{TeX Live}, etc.), o bien en línea empleando el servicio de edición \href{https://www.overleaf.com/latex/templates/plantilla-de-tfg-escuela-superior-de-informatica-uclm/phjgscmfqtsw}{Overleaf}.

Este documento se aprovecha para proporcionar información sobre la elaboración de la memoria del TFG con ayuda de \LaTeX{} empleando este documento como plantilla. Por este motivo, el documento sigue una estructura similar de secciones a la que debe presentar un TFG y muestras ejemplos de uso de distintos elementos y comandos de maquetación del documento.

\emph{Aunque la plantilla se ajusta a las necesidades y reglamentación de la ESI-UCLM, su adaptación es sencilla a otras titulaciones, instituciones y otros documentos de carácter académico. Esta plantilla permite de modo automático la elaboración de la memoria del documento en idioma inglés y puede ser utilizada en cualquier SO (Windows, Linux, Mac OSX, etc.).}


%---
\cleardoublepage % Se incluye para modificar el contador de página antes de añadir 





% EDITAR: Abstract (máx. 1 pág.)
%---
\phantomsection  % Necesario con hyperref
\addcontentsline{toc}{chapter}{Abstract} % Añade al TOC.
\selectlanguage{english} % Selección de idioma del resumen.
\makeatletter
\begin{center} %
   {\textsc{BACHELOR DISSERTATION - ESCUELA SUP. DE INFORMÁTICA (UCLM)}\par}
   \vspace{1cm} %  
   {\textbf{\Large Guided template for TFG}\par}
   \vspace{0.4cm} %
   {\@autor \\ \@cityTF,{} \@monthTF{} \@yearTF\par} 
   \vspace{0.9cm} %
   {\textbf{\large\textsf{Abstract}}\par} 
\end{center}   
\makeatother %
\emph{English version for the abstract.}
\cleardoublepage % Se incluye para modificar el contador de página antes de añadir 

